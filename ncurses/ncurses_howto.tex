%%
%% (
%%  )\ )                             (
%%  (()/(   (            (             )\  )   (
%%   /(_))  ))\   (       ))\  (   (   (()/(   ))\
%%   (_))  /((_)  )\  )  /((_) )\  )\   ((_))/((_)
%%   | _ \(_))(  _(_/( (_) )  ((_)((_)  _| |(_))
%%   |   /| || || ' \))/ -_)/ _|/ _ \/ _` |/ -_)
%%   |_|_\ \_,_||_||_| \___|\__|\___/\__,_|\___|
%%

\documentclass{article}
\usepackage[utf8x]{inputenc}
\usepackage{amsmath}
%\usepackage{slashbox}
\usepackage{amsfonts}
\usepackage{amssymb}
\usepackage{graphicx} % Paquete para incluir imágenes en el documento LaTeX
\usepackage{hyperref}
\hypersetup{
  colorlinks=true,
  linkcolor=blue,
  filecolor=magenta,
  urlcolor=cyan,
}
\urlstyle{same}
\usepackage{varwidth}

\newcommand\tab[1][1cm]{\hspace*{#1}}

\usepackage{multirow}

\usepackage[a4paper,rmargin=1.5cm,lmargin=1.5cm,top=1.5cm,bottom=1.5cm]{geometry}

\usepackage{pdfpages}

\usepackage{xcolor}
\usepackage{minted}
\setminted[cpp]{frame=lines, framesep=2mm, baselinestretch=1.2, rulecolor=\color{black!80},
                bgcolor=DarkGray,fontsize=\normalsize}
\usemintedstyle[cpp]{monokai}
\setminted[python3]{frame=lines, framesep=2mm, baselinestretch=1.2, rulecolor=\color{black!80}, bgcolor=DarkGray}
\usemintedstyle[python]{paraiso-dark}
\setminted[./pseudocode.py:PseudocodeLexer -x]{frame=lines, framesep=2mm, baselinestretch=1.2,
            rulecolor=\color{black!30}, bgcolor=LightGray, fontsize=\normalsize}
\usemintedstyle[./pseudocode.py:PseudocodeLexer -x]{rainbow_dash}
\setminted[bash]{baselinestretch=1.2,rulecolor=\color{black!30},bgcolor=LightGray}
\definecolor{LightGray}{gray}{0.98}
\definecolor{DarkGray}{gray}{0.1}
\definecolor{MidGray}{gray}{0.8}
\definecolor{codegreen}{rgb}{0,0.6,0}
\definecolor{codegray}{rgb}{0.5,0.5,0.5}
\definecolor{codepurple}{rgb}{0.58,0,0.82}
\definecolor{backcolour}{rgb}{0.95,0.95,0.92}

\setlength{\parindent}{0px}  % Setea la indentacion de la primera linea de cada parrafo a cero pixeles.


\title{Resolución de la novena semana}
\author{@RuneCode}

\begin{document}
%% Portada
\includepdf{./portada/portada.pdf}


%% ####################################################################################
%%    Inicio del Documento
%% ####################################################################################
\section{Introducción}%
En los viejos tiempos de los terminales de teletipo, los terminales estaban
lejos de las computadoras y se conectaban a ellos a través de cables seriales.
los terminales podrían ser configurados mediante el envío de una serie de
bytes. Todas las capacidades (tales como mover el cursor a una nueva ubicación,
borrar parte de la pantalla, el desplazamiento de pantalla, cambio de modos,
etc.) de los terminales se puede acceder a través de esta serie de bytes. Estas
secuencias de control generalmente se llaman secuencias de escape, ya que
comenzar con un carácter de escape(0x1B). Incluso hoy en día, con la adecuada
emulación, podemos enviar secuencias de escape a el emulador y lograr el mismo
efecto en una ventana de terminal.\\

Suponga que desea imprimir una línea en color. Pruebe a escribir esto en su
consola.\\

\mint{bash}|echo "^[[0;31;40mIn Color""]]"|

El primer carácter es un carácter de escape, que aparece como los caracteres $\wedge$
y $[$. Para poder imprimirlo, hay que pulsar CTRL+V y luego la tecla ESC. Todos
los demás son caracteres imprimibles normales. Usted debe ser capaz de ver la
cadena “In Color” en rojo. Se mantiene de esa forma y para volver de nuevo al
tipo de modo original.

\mint{bash}|echo "^[[0;37;40m""]]"|

Ahora, ¿qué significan estos caracteres mágicos? Difíciles de comprender?
Incluso podrían ser diferentes para diferentes terminales. Asi que los
diseñadores de UNIX inventaron un mecanismo denominado \textbf{termcap}. Es un
archivo que enumera todas las capacidades de un terminal en particular, junto
con las secuencias de escape necesarias para conseguir un efecto particular.\\
Años más tarde, este fue reemplazado por \textbf{terminfo}. Sin ahondar
demasiado en detalles, este mecanismo permite a los programas de aplicación
consultar la base de datos terminfo y obtener los caracteres de control para
ser enviados a un terminal o emulador de terminal.

\subsection{¿Qué es Ncurses?}%
Usted podría preguntarse, el significado de toda esta algarabia técnica. En el
escenario anterior, todos los programas de aplicación se supone consultan
terminfo y realizan las cosas necesarias (envío de caracteres de control,
etc.). Pronto se hizo difícil gestionar toda esta complejidad y esto dio a luz
a ‘CURSES’. Curses es un juego de palabras del nombre “cursor optimization”
(optimización del cursor). La libreria Curses forma un envoltorio para trabajar
encima de los códigos de terminal puros, y proporciona una API (Application
Programming Interface) altamente flexible y eficiente. Proporciona funciones
para mover el cursor, crear ventanas, producir colores, jugar con el ratón,
etc. Los programas de aplicación no necesitan preocuparse por las capacidades
subyacentes del terminal.\\

Entonces, ¿qué es NCURSES? NCURSES es un clon del original Sistem V Release 4.0
(SVr4) curses. Es una biblioteca de libre disposición, totalmente compatible
con la versión anterior de curses. En breve, es una libreria de funciones que
maneja la pantalla de una aplicación en terminales de celdas de caracter. En el
resto del documento, la términos curses y ncurses se utilizan indistintamente.\\

Una historia detallada de NCURSES se puede encontrar en el archivo NEWS dentro
del codigo fuente. El paquete actual es mantenido por
\href{to:dickey@his.com}{Thomas Dickey}. Puede contactar con los mantenedores
en \href{to:bug-ncurses@gnu.org}{bug-ncurses@gnu.org}.\\

\subsection{¿Qué podemos hacer con Ncurses?}%
Ncurses no sólo crea una envoltura sobre las capacidades del terminal, también
le da un marco sólido para crear una bonita UI (User Interface/interfaz de
usuario) en modo texto. Proporciona funciones para crear ventanas etc. Sus
bibliotecas hermanas \textbf{panel}, \textbf{menu} y \textbf{form} proporcionan
una extensión a la biblioteca básica curses. Estas bibliotecas por lo general
vienen tambien con Curses. Uno puede crear aplicaciones que contienen múltiples
ventanas, menús, paneles y formularios. Las ventanas se pueden manejar
independientemente, pueden proporcionar ‘deslizabilidad’ e incluso pueden
ocultarse.\\

Los menús ofrecen al usuario una fácil opción de selección de comandos. Los
formularios facilitan la entrada de datos y ventanas de visualización. Los
paneles extienden las capacidades de ncurses para hacer frente a ventanas
superpuestas y apiladas.\\

Estas son sólo algunas de las cosas básicas que podemos hacer con ncurses. A
medida que avancemos, veremos todas las capacidades de estas librerias.\\

\subsection{Objetivo / Alcance del documento}%
Este documento pretende ser una guía “Todo en Uno” para la programación con
ncurses y sus librerias hermanas. Nos graduamos desde un simple programa “Hola
Mundo” a una forma más compleja de manipulación. Se asume una nula experiencia
previa en ncurses. La escritura es informal, pero se proporciona una gran
cantidad de detalles de cada uno de los ejemplos.

\subsection{Acerca de los programas}%
Todos los programas en el documento están disponibles en forma de empaquetado
\href{http://www.tldp.org/HOWTO/NCURSES-Programming-HOWTO/ncurses_programs.tar.gz}{aquí}.
Descomprima y desempaquete.\\

Hay un Makefile de nivel superior en el directorio principal. Este construye
todos los archivos y pone los ejecutables listos para su uso en el directorio
demo/exe. También puede construir los programas selectivamente mediante el make
del directorio correspondiente. Cada directorio contiene un archivo README
explicando el propósito de cada archivo c en el directorio.\\

Para cada ejemplo, he incluido el nombre de ruta relativo al directorio de
ejemplos.\\

Todos los programas son liberados bajo la misma licencia usada para ncurses
(estilo MIT). Esto le da la capacidad de hacer cualquier cosa ademas de
reclamar todo como suyo. Siéntase libre de utilizarlos en sus programas, según
convenga.\\

\subsection{Otros formatos del documento}%
Esta guía también está disponible en otros formatos varios en el sitio
tldp.org. Aquí están los enlaces a otros formatos de este documento.
\subsubsection{Formato de fácil acceso desde tldp.org}%
\begin{itemize}
  \item \href{http://www.ibiblio.org/pub/Linux/docs/HOWTO/other-formats/pdf/NCURSES-Programming-HOWTO.pdf}{Acrobat PDF Format}
  \item \href{http://www.ibiblio.org/pub/Linux/docs/HOWTO/other-formats/ps/NCURSES-Programming-HOWTO.ps.gz}{PostScript Format}
  \item \href{http://www.ibiblio.org/pub/Linux/docs/HOWTO/other-formats/html/NCURSES-Programming-HOWTO-html.tar.gz}{In multiple HTML pages}
  \item \href{http://www.ibiblio.org/pub/Linux/docs/HOWTO/other-formats/html_single/NCURSES-Programming-HOWTO.html}{In One big HTML format}
\end{itemize}

\subsection{Créditos}%
Doy las gracias a \href{sharath_1@usa.net}{Sharath} y Emre Akbas por ayudarme
con algunas secciones. La introducción fue escrita inicialmente por sharath. La
Reescribí con algunos extractos tomados de su trabajo inicial. Emre ayudó a
escribir las secciones printw y scanw.\\

Los equivalentes Perl de los programas de ejemplo son aportados por
\href{Aratnaweera@virtusa.com}{Anuradha Ratnaweera}.\\

Luego viene \href{parimi@ece.arizona.edu}{Ravi Parimi}, mi querido amigo, que
ha estado en este proyecto antes de escribir ni una línea. Constantemente me ha
bombardeado con sugerencias y ha revisado pacientemente todo el texto. Él
también comprobó cada programa en Linux y Solaris.\\

\subsection{Derechos de Autor}%
Copyright {\copyright} 2001 by Pradeep Padala.\\

Permission is hereby granted, free of charge, to any person obtaining a copy of
this software and associated documentation files (the “Software”), to deal in
the Software without restriction, including without limitation the rights to
use, copy, modify, merge, publish, distribute, distribute with modifications,
sublicense, and/or sell copies of the Software, and to permit persons to whom
the Software is furnished to do so, subject to the following conditions:\\

The above copyright notice and this permission notice shall be included in all
copies or substantial portions of the Software.\\

THE SOFTWARE IS PROVIDED “AS IS”, WITHOUT WARRANTY OF ANY KIND, EXPRESS OR
IMPLIED, INCLUDING BUT NOT LIMITED TO THE WARRANTIES OF MERCHANTABILITY,
FITNESS FOR A PARTICULAR PURPOSE AND NONINFRINGEMENT. IN NO EVENT SHALL THE
ABOVE COPYRIGHT HOLDERS BE LIABLE FOR ANY CLAIM, DAMAGES OR OTHER LIABILITY,
WHETHER IN AN ACTION OF CONTRACT, TORT OR OTHERWISE, ARISING FROM, OUT OF OR IN
CONNECTION WITH THE SOFTWARE OR THE USE OR OTHER DEALINGS IN THE SOFTWARE.\\

Except as contained in this notice, the name(s) of the above copyright holders
shall not be used in advertising or otherwise to promote the sale, use or other
dealings in this Software without prior written authorization.\\

\section{Hola Mundo!!!}%
Bienvenido al mundo de curses. Antes de sumergimos en la libreria y mirar en
sus diversas características, vamos a escribir un programa simple y decir hola
al mundo.\\

\subsection{Compilar con la librería NCURSES}%
Para utilizar las funciones de la libreria ncurses, usted tiene que incluir
ncurses.h en sus programas. Para enlazar el programa con ncurses la bandera
-lncurses debe ser agregada.

\begin{minted}{cpp}
  #include <ncurses.h>
  .
  .
  .
\end{minted}

Compilacion y enlace: gcc $<$archivo de programa$>$ -lncurses\\

%C++
\href{https://github.com/nasciiboy/NCURSES-Programming-HOWTO/blob/master/ncurses_programs/basics/hello_world.c}{*Ejemplo 1. El programa Hello World!!!}
\inputminted{cpp}{./cpp/001_hello_world.cpp}

\subsection{Disección}%
El programa anterior inmprime “Hello World !!!” a la pantalla y termina. Este
programa muestra cómo inicializar curses, manipular la pantalla y finalizar el
modo curses. Vamos a diseccionar línea por línea.\\

\subsubsection{Acerca de initscr()}%
La funcion initscr() inicializa el terminal en modo de curses. En algunas
implementaciones, se borra la pantalla y presenta una pantalla en blanco. Para
cualquier manipulación de pantalla utilizando el paquete curses este ha de ser
llamado primero. Esta función inicializa el sistema curses y asigna memoria
para la ventana actual (llamanda \textbf{stdscr}) y algunas otras estructuras
de datos. En casos extremos esta función puede fallar por falta de memoria
para almacenar las estructuras de datos de la libreria curses.\\

Una vez hecho esto, podemos hacer una variedad de inicializaciones para
personalizar nuestros ajustes a curses. Estos detalles serán explicados. Más
adelante.\\

\subsubsection{La misteriosa refresh()}%
La siguiente línea printw imprime la cadena “Hello World !!!” en el pantalla.
Esta función es análoga al printf normal en todos aspectos excepto que imprime
los datos en una ventana llamada stdscr en las coordenadas (y,x) actuales. Al
inicio nuestras coordenadas actuales están en 0,0 la cadena se imprime en la
esquina izquierda de la ventana.\\

Esto nos lleva al misterioso refresh(). Cuando llamamos a printw los datos se
escriben realmente en una ventana imaginaria, esto no actualiza la pantalla
todavía. El trabajo de printw es actualizar algunas banderas y estructuras de
datos y escribir los datos a el buffer correspondiente a stdscr. Con el fin de
mostrar esto en pantalla, deberá llamar a refresh() y decirle al sistema de
curses que debe volcar el contenido en pantalla.\\

La filosofía detrás de todo esto es permitir al programador hacer varias
actualizaciones en la pantalla o ventana imaginaria y hacer un refresco una vez
terminada toda la actualización de pantalla. refresh() comprueba la ventana y
actualiza sólo la porción que ha cambiado. Esto mejora el rendimiento y también
ofrece una mayor flexibilidad. Pero a veces es frustrante para los
principiantes. Un error común cometido por los principiantes es olvidar llamar
a refresh() después de que hacer algunas actualizacions mediante funciones de
clase printw(). A veces todavía lo olvido añadir :-)\\

\subsubsection{Acerca de endwin()}%
Y, finalmente, no se olvide poner fin al modo curses. De lo contrario su
terminal puede comportarse de forma extraña después de cerrar el programa.
endwin() libera la memoria tomada por el sub-sistema de curses y sus
estructuras de datos y pone el terminal en modo normal. Esta función debe ser
llamada después de que haya terminado con el modo de curses.

\section{Los detalles sangrientos}%
Ahora que hemos visto cómo escribir un programa sencillo vamos a entrar en
detalles. Hay muchas funciones que ayudan a personalizar lo que se ve en
pantalla y muchas características que pueden tener un uso completo.\\

Aquí vamos…

\section{Inicialización $<<$@INIT$>>$}%
Ahora sabemos que para inicializar el sistema curses la funcion initscr() ha de
ser llamada. Hay funciones que pueden ser llamadas después de esta
inicialización para personalizar nuestra sesión de curses. Podemos cuestionar
al sistema curses para configurar el terminal en modo raw o inicializar el
color o el ratón, etc$\cdots$ Vamos a discutir algunas de las funciones que
normalmente se llaman inmediatamente después de initscr();

\subsection{Funciones de inicialización}%
\subsubsection{raw() and cbreak()}%
Normalmente, el terminal maneja buffers de caracteres que el usuario escribe
hasta que se encuentra una línea nueva o retorno de carro. Pero la mayoria de
programas requiere que los caracteres estén disponibles tan pronto como los
escribe el usuario. Las dos funciones anteriores se utilizan para desactivar el
buffer de línea. La diferencia entre estas dos funciones es la forma en que los
caracteres de control como suspender (CTRL-Z), interrumpir y quitar (CTRL-C) se
pasan al programa. En el modo raw() los caracteres pasan directamente al
programa sin generar una señal. En el modo cbreak() estos caracteres de control
son interpretados como cualquier otro carácter por el controlador de terminal.
Personalmente prefiero usar raw() ya que puedo ejercer un mayor control de lo
que hace el usuario.

\subsubsection{echo() and noecho()}%
Estas funciones controlan el eco de caracteres generados por el usuario al
terminal. noecho() desactiva el eco. La posible razón para hacer esto es
mejorar el control sobre el eco o suprimir el eco innecesario al tomar la
entrada del usuario mediante funciones getch(). La mayor parte de los programas
interactivos llaman a noecho() en la inicialización y realizan el eco de
caracteres de manera controlada. Esto le da al programador la flexibilidad de
imprimir caracteres en cualquier lugar en la ventana sin actualizar las
coordenadas actuales (y, x).

\subsubsection{keypad()}%
Este es mi función de inicialización favorita. Habilita la lectura de teclas de
función como F1, F2, teclas de direccion, etc. Casi todos los programa
interactivos permiten esto, Las teclas de dirección son una parte importante de
cualquier interfaz de usuario. Coloque keypad(stdscr, TRUE) para habilitar esta
función en la pantalla (stdscr). Usted aprenderá más sobre gestión de teclas en
secciones posteriores de este documento.

\subsubsection{halfdelay()}%
Esta función, aunque no se utiliza muy a menudo, en ocaciones suele ser de
utilidad. halfdelay() es llamada para activar el modo de espera medio, que es
similar al modo cbreak() en el que los caracteres escritos estan disponibles de
inmediato para el programa. Sin embargo, espera ‘X’ décimas de segundo por la
entrada y luego devuelve ERR, si no hay entrada disponible. ‘X’ es el valor de
tiempo de espera pasado a la función halfdelay(). Esta función es útil cuando
se pide entrada al usuario, y si no responde en cierto tiempo, podemos hacer
algo más. Un ejemplo posible es un tiempo de espera en la solicitud de
contraseña.

\subsection{Otras funciones de inicialización}%
Hay algunas funciones más que se llaman en la inicialización para personalizar
el comportamiento de curses. No se utilizan tan ampliamente como las
mencionadas anteriormente. Algunas de estas se explican donde es apropiado.

\subsection{Un ejemplo}%
Vamos a escribir un programa que permitirá aclarar el uso de éstas funciones.\\

%C++
\href{https://github.com/nasciiboy/NCURSES-Programming-HOWTO/blob/master/ncurses_programs/basics/init_func_example.c}{*Ejemplo 2. Funciones de inicialización} 
\inputminted{cpp}{./cpp/002_funciones_inicializacion.cpp}

Este programa se explica por sí mismo. Pero hago uso de funciones que aun no se
han explicado. La función getch() se utiliza para obtener un caracter del
usuario. Es equivalente un getchar() normal excepto que puede deshabilitar el
buffer de línea para evitar <enter> después de la entrada. Para más información
sobre getch() y lectura de teclas eche un vistazo a la seccion gestion de
teclado. Las funciones attron y attroff se utilizan para cambiar y desactivar
algunos atributos respectivamente. En el ejemplo son usadas para imprimir el
carácter en negrita. Estas funciones se explican en detalle más adelante.

\section{Una indicación acerca de las Ventanas}%
Antes de sumergirse en las innumerables funciones de ncurses, permítanme
aclarar algunas cosas acerca de las ventanas. Las ventanas se explican en
detalle en la siguientes secciones.\\

Una ventana es una pantalla imaginaria definida por el sistema de curses. una
ventana no significa una ventana bordeada que por lo general se ven en
plataformas Win9X. Cuando inicializa curses, se crea una ventana por defecto
llamada stdscr que representa su pantalla de 80x25 (o el tamaño de ventana en
el que se está ejecutando). Si está realizando tareas simples como imprimir
algunas cadenas, lectura de entrada, etc., puede utilizar con seguridad esta
única ventana para todos los propósitos. También se pueden crear ventanas y
llamar a funciones que trabajan de forma explícita en una ventana especificada.\\

Por ejemplo, si usted llama a.

\begin{minted}{cpp}
  printw("Hola ahi!!!");
  refresh();
\end{minted}

Imprime la cadena en stdscr en la posicion actual del cursor. Del mismo modo la
llamada a refresh(), funciona sólo en stdscr.\\

Digamos que ha creado ventanas entonces tiene que llamar a una función con una
‘w’ añadida a la función habitual.

\begin{minted}{cpp}
  wprintw(win, "Hola ahi!!!");
  wrefresh(win);
\end{minted}

Como se verá en el resto del documento, Los nombres de funcione siguen la misma
convención. Para cada función por lo general hay tres funciones más.

\begin{minted}{cpp}
  printw(string);         /* Imprime en stdscr en la posicion actual del cursor */
  mvprintw(y, x, string); /* Mueve a (y, x) luego imprime la cadena */
  wprintw(win, string);   /* Imprime en la ventana win en la actual posicion del cursor
                           * en la ventana */
  mvwprintw(win, y, x, string); /* Mueve a las coordenadas (y, x) relativas a la
                                 * ventana y luego imprime */
\end{minted}

Por lo general, las funciones sin-w son macros que se expanden a la
correspondiente funcion-w con stdscr como parámetro de ventana.

\section{Funciones de Salida}%
Supongo que no puede esperar más para ver algo de acción. De nuevo a nuestra
odisea de funciones. Ahora que curses esta inicializado, vamos a interactuar
con el mundo.\\

Hay tres clases de funciones que se pueden utilizar para manejar la salida en
la pantalla.\\

\begin{itemize}
  \item Clase addch(): Imprimir carácter individual con atributos
  \item Clase printw(): Imprimir salida con formato similar a printf()
  \item Clase addstr(): Imprimir cadenas
\end{itemize}

Estas funciones se pueden utilizar indistintamente y es cuestión de estilo
elegir la clase a utilizar. Veamos cada una a detalle.

\subsection{Clase de funciones addch()}%
Estas funciones colocan un solo carácter en la posicion del cursor y avanzan
la posición del cursor. Puede dar el carácter a imprimir pero por lo general se
utilizan para imprimir un caracter con algunos atributos. Los atributos se
explican a detalle en posteriores secciones del documento. Si un caracter está
asociado a un atributo (negrita, vídeo inverso, etc.), cuando curses imprime el
carácter, este se imprime con ese atributo.\\

A fin de combinar un caracter con algunos atributos, tiene dos Opciones:\\

- Con el operador OR enmascarar un solo caracter y la macro del atributo deseado.
Estas macros de atributos se encuentran en el archivo de cabecera ncurses.h.
Por ejemplo, si desea imprimir un carácter ch (de tipo char) negrita y
subrayado, llamaría a addch() de la siguiente manera.

\begin{minted}{cpp}
  addch(ch | A_BOLD | A_UNDERLINE);
\end{minted}

- Mediante funciones como attrset(), attron(), attroff(). Estas funciones se
explican en la sección Attributos. Brevemente, se pueden manipular los
atributos actuales de una ventana dada. Una vez establecidos, el carácter
impreso está asociado con los atributos hasta que se desactivan.\\

Adicionalmente, curses proporciona algunos caracteres especiales para gráficos
basados en caracteres. Puede dibujar tablas, líneas horizontales o
verticales, etc. Puede encontrar todos los caracteres disponibles en el archivo
de cabecera ncurses.h. Intente buscando macros que inicien con ACS\_ en dicho
archivo.

\subsection{mvaddch(), waddch(), mvwaddch()}%
\textbf{mvaddch()} se utiliza para mover el cursor a un punto dado, y luego imprimir.
Por tanto, las llamadas:mvaddch() se utiliza para mover el cursor a un punto
dado, y luego imprimir. Por tanto, las llamadas:

\begin{minted}{cpp}
  move(row, col);       /* Mueve el cursor a la enesima fila y enesima columna */
  addch(ch);
\end{minted}

Puede ser reemplazado por

\begin{minted}{cpp}
  mvaddch(row, col, ch);
\end{minted}

\textbf{waddch()} es similar a addch(), excepto que añade un carácter a una ventana
dada. (Tenga en cuenta que addch() añade un carácter a la ventana stdscr.)\\

De manera similar la función \textbf{mvwaddch()} se utiliza para añadir un carácter a la
ventana dada en las coordenadas dadas.\\

Ahora, estamos familiarizados con la función de salida basico addch(). Pero, si
queremos imprimir una cadena, sera muy molesto imprimirla carácter a carácter.
Afortunadamente, ncurses provee funciones similares a printf y puts.

\subsection{Clase de funciones printw()}%
Estas funciones son similares a printf() con la capacidad añadida de imprimir
en cualquier posición de la pantalla.

\subsubsection{printw() y mvprintw()}%
Estas dos funciones operan similar a printf(). \textbf{mvprintw()} se utiliza para mover
el cursor a una posición y luego imprimir. Si quiere puede mover el cursor y
luego imprimir utilizando la función printw(), primero utilice move() y luego
printw() aunque no veo ningún punto para evitar el uso de mvprintw(), la
flexibilidad de manejo es tuya.

\subsubsection{wprintw() y mvwprintw()}%
Estas dos funciones son similares a las de dos de arriba, excepto que imprimen
en la correspondiente ventana dada como argumento.

\subsubsection{vwprintw()}%
Esta función es similar a vprintf(). Se puede utilizar para imprimir un número
variable de argumentos.

\subsubsection{Un ejemplo sencillo de printw}%

%C++
\href{https://github.com/nasciiboy/NCURSES-Programming-HOWTO/blob/master/ncurses_programs/basics/printw_example.c}{*Ejemplo 3. Un ejemplo sencillo de printw}
\inputminted{cpp}{./cpp/003_printw.cpp}

El Programa anterior demuestra lo fácil que es utilizar printw. Simplemente
pase las coordenadas y el mensaje a mostrar en pantalla, y luego haga lo que
quiera.\\

El programa anterior introduce una nueva función getmaxyx(), una macro definida
en ncurses.h. Entrega el número de columnas y el número de filas en una ventana
dada. getmaxyx() hace esto actulalizando las variables que le son dadas. Ya que
getmaxyx() no es una función no pase punteros en ella, solo entregue dos
variables entero.

\subsection{Clase de funciones addstr()}%
\textbf{addstr()} se utiliza para poner una cadena de caracteres en una ventana
determinada. Esta función es similar a llamar a addch() una vez por cada
carácter de una cadena dada. Esto es cierto para todas las funciones de salida.
Hay otras funciones de esta familia como mvaddstr(), mvwaddstr() y waddstr(),
que obedecen la convención de nomenclatura de curses. (por ejemplo mvaddstr()
es similar a las respectivas llamadas move() y luego addstr().) Otra de las
funciones de esta familia es addnstr(), que adicionalmente toma un parámetro
entero (llamado n). Esta función pone almenos n caracteres en el pantalla. Si n
es negativo, entonces se añadirá toda la cadena.

\subsection{Una palabra de advertencia}%
Todas estas funciones toman en sus argumentos la coordenada y primero y x
despues. Un error común de los principiantes es pasar x, y en ese orden. Si
está haciendo demasiadas manipulaciones de coordenadas (y, x), considere
dividir la pantalla en ventanas y manipular cada una por separado. Las ventanas
se explican en la sección Ventanas.

\section{Funciones de entrada}%
Bueno, la impresión sin toma de entrada, es aburrida. Veamos las funciones que
nos permiten tomar la entrada de usuario. Estas funciones también pueden ser
divididas en tres categorías.

\begin{itemize}
  \item Clase getch(): Obtener un caracter
  \item Clase scanw(): Obtener entrada con formato
  \item Clase getstr(): Obtener cadenas
\end{itemize}

\subsection{Clase de funciones getch()}%
Estas funciones leen un solo carácter de la terminal. Pero hay varios hechos
sutiles a considerar. Por ejemplo, si no utiliza la función cbreak(), curses no
leerán sus caracteres de entrada forma continua, solo empezara a leerlos sólo
después de encontrar un caracter de nueva linea o un EOF. Con el fin de evitar
esto, debe utilizarse la funcion cbreak() para que los caracteres están
disponibles a su prograba inmediatamente. Otra función muy utilizada es
noecho(). Como el nombre sugiere, cuando esta función es activada (usada), los
caracteres pulsados por el usuario no se mostrarán en la pantalla. Las dos
funciones cbreak() y noecho() son ejemplos típicos de gestión del teclado. Las
funciones de este género se explican en la sección gestion del teclado.

\subsection{clase de funciones scanw()}%
Estas funciones son similares a scanf() con la capacidad añadida de conseguir
la entrada desde cualquier lugar de la pantalla.

\subsubsection{scanw() y mvscanw}%
El uso de estas funciones es similar al de sscanf(), donde la línea a escanear
es proporcionada por la funcien wgetstr(). Es decir, estas funciones llaman a
la funcion wgetstr() (explicada más adelante) y utilizan la línea resultante de
un escaneo.

\subsubsection{wscanw() y mvwscanw()}%
Estas son similares a las dos funciones anteriores excepto que leen desde una
ventana, que se suministra como uno de los argumentos a estas funciones.

\subsubsection{vwscanw()}%
Esta función es similar a vscanf(). Se puede utilizar cuando se va a escanear
una número variable de argumentos.

\subsection{Clase de funciones getstr()}%
Estas funciones se utilizan para obtener cadenas desde la terminal.
Esencialmente, esta función realiza la misma tarea que la que se lograría con
una serie de llamadas a getch() hasta recibir nueva línea, retorno de carro, o
fin de archivo. La cadena de caracteres resultante es apuntada por str, El cual
es un puntero a carácter proporcionado por el usuario.

\subsection{Algunos ejemplos}%

%C++
\href{https://github.com/nasciiboy/NCURSES-Programming-HOWTO/blob/master/ncurses_programs/basics/scanw_example.c}{*Ejemplo 4. Un ejemplo sencillo de scanw*}
\inputminted{cpp}{./cpp/004_scanw.cpp}

\section{Atributos $<<$@ATTRIB$>>$}%
Hemos visto un ejemplo de cómo se pueden utilizar atributos para imprimir
caracteres con algunos efectos especiales. Cuando los atributos se establecen
con prudencia, pueden presentar información de manera sencilla y comprensible.
El siguiente programa toma un archivo C como entrada e imprime el archivo con
comentarios en negrita. Escanea a través del código.\\

%C++
\href{https://github.com/nasciiboy/NCURSES-Programming-HOWTO/blob/master/ncurses_programs/basics/simple_attr.c}{*Ejemplo 5. Un ejemplo sencillo de atributos*}
\inputminted{cpp}{./cpp/005_atributos.cpp}

No se inquiete con toda esa inicialización y otras mierdas. Concéntrese en el
bucle while. Se lee cada carácter en el archivo y se busca el patrón /*. Una
vez visto el patrón, cambia el atributo a BOLD con attron(). Cuando se obtiene
el patrón */ se desactiva mediante attroff().\\

El programa anterior también nos presenta dos funciones útiles getyx() y
move(). La primera función obtiene las coordenadas actuales del cursor en las
variables y, x. Ya que getyx() es una macro no tenemos que pasar punteros a
variables. La función move() mueve el cursor a las coordenadas que le son
dadas.\\

El programa anterior es realmente muy simple ya que no hace mucho. Con estas
líneas se podría escribir un programa más útil que lea un archivo C, lo analice
e imprima en diferentes colores. Podría incluso extenderlo a otros lenguajes.\\

\subsection{Los detalles}%
Entremos en más detalles sobre los atributos. Las funciones attron(),
attroff(), attrset(), y sus funciones hermanas attr\_get() etc… se utilizan
para encender/apagar atributos, obtener atributos y dar una presetracion
colorida.\\

Las funciones attron y attroff toman una máscara de bits de atributos para
encenderlos o apagarlos, respectivamente. Los siguientes atributos de video,
definidos en $<$curses.h$>$ pueden ser pasados a estas funciones.

\begin{itemize}
  \item A\_NORMAL \tab Visualizacion normal (sin resaltar)
  \item A\_STANDOUT \tab Mejor modo del terminal para resaltado
  \item A\_UNDERLINE \tab Subrayado
  \item A\_REVERSE \tab Video inverso
  \item A\_BLINK \tab Intermitente
  \item A\_DIM \tab Mitad de brillo
  \item A\_BOLD \tab Brillo extra o negrita
  \item A\_PROTECT \tab Modo protegido
  \item A\_INVIS \tab En modo Invisible o en blanco
  \item A\_ALTCHARSET \tab Juego de caracteres alternativo
  \item A\_CHARTEXT \tab Mascara de bits para extraer el caracter
  \item COLOR\_PAIR(n) \tab Par de color numero /n/
\end{itemize}

El último es el más colorido :-) Los colores se explican en las secciones
siguentes.\\

Con OR(|) podemos poner cualquier número de los atributos anteriores para
obtener un efecto combinado. Si querías vídeo inverso con caracteres
parpadeantes puedes usar.\\


\begin{minted}{cpp}
  attron(A_REVERSE | A_BLINK);
\end{minted}

\subsection{attron() vs attrset()}%
Entonces ¿cuál es la diferencia entre attron() y attrset()? attrset establece
los atributos de la ventana mientras que attron simplemente enciende el
atributo que le es dado. Así attrset() anula totalmente los atributos que la
ventana tenía anteriormente y establece nuevos atributos. Del mismo modo
attroff() simplemente se apaga el atributo que se le da como argumento. Esto
nos da la flexibilidad de gestiónar los atributos facilmente. Pero si se
utilizan descuidadamente puede perder la pista de que atributos tiene la
ventana y distorsionar la pantalla. Esto es especialmente cierto cuando maneja
menús con colores y resaltado. Así que determine una política constante y
adhierase a ella. Siempre puede utilizar standend() (que es equivalente a
attrset(A\_NORMAL)) para apagar todos los atributos y regresar al modo normal.

\subsection{attr\_get()}%
La función attr\_get() obtiene los atributos y colores actuales de la ventana.
Aunque tal vez la utilice tan a menudo como las funciones anteriores, es útil
para escanear áreas de la pantalla. Digamos que queríamos realizar alguna
actualización compleja en la pantalla y no estamos seguros de cuales atributos
estan asociados a cada carácter. En ese caso puede utilizadar esta función en
conjunto con attrset o attron para producir el efecto deseado.

\subsection{Funciones attr\_}%
Hay una serie de funciones como attr\_set(), attr\_on, etc.. Estas son similares
a las funciones anteriores, excepto que toman parámetros de tipo attr\_t.

\subsection{funciones wattr}%
Para cada una de las funciones anteriores tenemos una función correspondiente
con ‘w’, que opera en una ventana en particular. Las funciones anteriores
operar en stdscr.

\subsection{funciones chgat()}%
La función chgat() aparece al final de la página man curs\_attr. En realidad, es
una utilidad. Puede utilizar esta función para establecer los atributos de un
grupo de caracteres sin moverse. Con ello quiero decir!!! sin mover el cursor
:-) Esto cambia los atributos de un determinado número de caracteres a partir
de la ubicación actual del cursor.\\

Pasando -1 como numero de caracteres actualiza hasta el final de línea. Si
desea cambiar los atributos a los caracteres de la posición actual hasta el fin
de línea, sólo utilice:

\begin{minted}{cpp}
  chgat(-1, A_REVERSE, 0, NULL);
\end{minted}

Esta función es útil cuando se cambian los atributos de caracteres que ya estan
en pantalla. Valla al caracter desde el que desea cambiar y modifique el
atributo.\\

Las otras funciones wchgat(), mvchgat(), mvwchgat() se comportan de manera
similar excepto que las funciones “w” operan en un ventana concreta. Las
funciones mv primero mueven el cursor y a continuación realizan el trabajo
dado. chgat en realidad es una macro que se sustituye por wchgat() con stdscr
como ventana. La mayor parte de las funciones de sin “w” son macros.\\

\href{https://github.com/nasciiboy/NCURSES-Programming-HOWTO/blob/master/ncurses_programs/basics/with_chgat.c}{*Ejemplo 6. Ejemplo de uso para chgat()*}
\inputminted{cpp}{./cpp/006_chgat.cpp}

Este ejemplo también nos introduce en el mundo de color de curses. Los colores
serán explicadas en detalle más adelante. Utilice 0 para ningún color.

\section{Ventanas $<<$@WINDOWS$>>$}%
Las ventanas forman el concepto más importante en curses. Has visto la ventana
estándar stdscr en la cual todas funciones anteriores operan de forma
implícita. Ahora para hacer el diseño incluso de una GUI sencilla, recurriremos
al uso de ventanas. La principal razón para utilizar ventanas es para manipular
partes de la pantalla por separado, abtenietdo asi un mejor redimiento,
actualizando sólo las ventanas que necesitan cambiar y para un diseño mejor. Yo
diría que la última razón es la más importante para utilizar ventanas. Usted
siempre debe luchar por un diseño de sus programas mejor y más fácil de
gestionar. Si esta escribiendo grandes y complejas GUIs esto es de fundamental
importancia antes de empezar a hacer cualquier cosa.

\subsection{Lo basico}%
Puede crear una ventana llamando a la función newwin(). Esto en realidad no
crea nada en la pantalla. Asigna memoria a una estructura para manipular la
ventana y actualiza la estructura con los datos relativos a la ventana como su
tamaño, inicioy, iniciox etc.. De ahí que en curses, una ventana sea sólo una
abstracción imaginaria de una ventana, que se puede manipular independiente de
las demás partes de la pantalla. La funcion newwin() devuelve un puntero a la
estructura WINDOW, el cual puede pasar a funciones relacionadas con ventanas
como wprintw() etc.. Finalmente la ventana puede ser destruida con delwin().
Esto libera la memoria asociada con la estructura WINDOW.

\subsection{Que haya una Ventana!!!}%
Lo divertido es, crear una ventana y no poder verla. Comencemos mostrando la
ventana. La función box() sirve para dibujar un borde alrededor de la ventana.
Vamos a explorar esta y otras funciones con más detalle en este ejemplo.\\

\href{https://github.com/nasciiboy/NCURSES-Programming-HOWTO/blob/master/ncurses_programs/basics/win_border.c}{*Ejemplo 7. Borde en ventana}
\inputminted{cpp}{./cpp/007_borde_en_ventana.cpp}

\subsection{Explicación}%
No grites. Sé que es un ejemplo grande. Pero tengo que explicar algunas cosas
importantes aquí :-). Este programa crea una ventana rectangular que se puede
mover con las teclas de flecha izquierda, derecha, arriba y abajo. Con la
creacion y destruccion repetida de ventanas cuando el usuario presiona una
tecla. No valla más allá de los límites de la pantalla. La comprobación de
estos límites se deja como un ejercicio para el lector. Vamos a diseccionar
línea por línea.\\

El funcion create\_newwin() crea una ventana con newwin() y muestra un borde
alrededor con box(). La función destroy\_win() primero borra la ventana de
pantalla pintando un borde con el caracter ’ ’ y luego llama a delwin() para
liberar la memoria. Dependiendo de la tecla que el usuario presiona, se cambia
starty o startx y se crea una nueva ventana.\\

Como puede ver en destroy\_win, he usado wborder() en lugar de box(). La razón
está escrita en los comentarios (Te lo perdiste. Lo se. Lee el codigo :-)).
wborder() dibuja un borde alrededor de la ventana con los caracteres dados para
las 4 esquinas y las 4 líneas. Para decirlo claramente, si llama a wborder()
como a continuación:

\begin{minted}{cpp}
  wborder(win, '|', '|', '-', '-', '+', '+', '+', '+');
\end{minted}

\subsection{Las otras cosas en el ejemplo}%
También puede ver que en los ejemplos anteriores, he utilizado las variables
COLS y LINES que se inicializan con el tamaño de pantalla después de llamar a
initscr(). Pueden ser útiles para obtener las dimensiones de la pantalla y para
localizar las cordenadas al centro de la pantalla. Como siempre la función
getch() obtiene informacion del teclado y de acuerdo a esta se hace el trabajo
correspondiente. Este tipo de switch-case es muy común en todos los programas
basados en GUI.

\subsection{Otras funciones de Borde}%
El programa anterior es muy ineficiente ya que a cada pulsación, se destruye
una ventana y se crea otra. Así que vamos a escribir un programa más eficiente
que utiliza otras funciones relacionadas con bordes.\\

El siguiente programa utiliza mvhline() y mvvline() para lograr un efecto
similar. Estas dos funciones son simples. Crean un línea horizontal o vertical
de la longitud especificada en la posición especificada.\\

\href{https://github.com/nasciiboy/NCURSES-Programming-HOWTO/blob/master/ncurses_programs/basics/other_border.c}{Ejemplo 8. Más funciones de borde}
\inputminted{cpp}{./cpp/008_borde.cpp}

\section{Colores $<<$@color$>>$}%
\subsection{Lo basico}%
La vida parece aburrida sin colores. curses tiene un buen mecanismo para el
manejo de colores. Vamos a entrar de lleno con un pequeño programa.\\

\href{https://github.com/nasciiboy/NCURSES-Programming-HOWTO/blob/master/ncurses_programs/basics/simple_color.c}{*Ejemplo 9. Un ejemplo simpre de color*}
\inputminted{cpp}{./cpp/009_color.cpp}

Como puede ver, para empezar a utilizar el color, primero debe llamar a la
función start\_color(). Después de eso, puede utilizar las capacidades de color
de su terminal utilizando distintas funciones. Pasa averiguar si un terminal
tiene capacidades de color o no, puede utilizar la funcion has\_colors(), que
devuelve FALSE si el terminal no soporta color.\\

curses inicializa todos los colores admitidos por el terminal cuando llama a
start\_color(). Estos estan accesibles mediante constantes como COLOR\_BLACK,
etc. Ahora para realizar un manejo efectivo de colores, debe definir pares. Los
colores siempre se utilizan en pares. Esto significa hacer uso de la función
init\_pair() para definir el color de el frente y el fondo para un numero de
pares deseado. Después los pares pueden ser utilizados como un atributo normal
con la funcion COLOR\_PAIR(). Esto puede parecer incómodo al inicio. Pero esta
elegante solución nos permite gestionar pares de colores muy facilmente. Para
apreciarlo, mire el codigo fuente de “dialog”, una utilidad para mostrar
cuadros de diálogo en scripts de shell. Los desarrolladores definen todas las
combinaciones de color que se podiran necesitar y son inicializas al inicio.
Esto fácilita establecer atributos y acceder a parejas ya definidas como
constantes.\\

Los siguientes colores están definidos en curses.h. Usted puede hacer uso de
estos como parámetros para las distintas funciones de color.\\

\begin{itemize}
  \item COLOR\_BLACK   0
  \item COLOR\_RED     1
  \item COLOR\_GREEN   2
  \item COLOR\_YELLOW  3
  \item COLOR\_BLUE    4
  \item COLOR\_MAGENTA 5
  \item COLOR\_CYAN    6
  \item COLOR\_WHITE   7
\end{itemize}

\subsection{Cambio en Definiciones de Color}%
Puede utilizar la función init\_color() para cambiar los valores RGB definidos
por curses inicialmente. Digamos que quiere aligerar la intensidad del color
rojo para una minúscula. Entonces puede utilizar esta función como

\begin{minted}{cpp}
  init_color(COLOR_RED, 700, 0, 0);
  /* param 1     : nombre de color
  * param 2, 3, 4 : contenido rgb min = 0, max = 1000 */
\end{minted}

Si el terminal no puede cambiar las definiciones de color, la función devuelve
ERR. Puede utilizar la función can\_change\_color() para conocer si el terminal
tiene la capacidad de cambiar el contenido de color o no. El contenido rgb se
escala de 0 a 1000. Inicialmente COLOR\_RED es definido con un contenido de
1,000(r), 0(g), 0(b).

\subsection{Contenido en color}%
Puede utilizar las funciones color\_content() y pair\_content() para conocer
el contenido de color y la combinacion de frente y fondo de un par.

\section{Comunicandonos con el teclado $<<$@KEYS$>>$}%
\subsection{Lo basico}%
Ningun GUI está completo sin una fuerte interfaz de usuario y para interactuar
con el usuario, un programa curses debe ser sensible a la pulsacion de teclas o
las acciones del ratón realizadas por el usuario. Vamos a tratar con las teclas
primero.\\

Como se ha visto en casi todos los ejemplos anteriores, es muy fácil obtener la
entrada de teclado del usuario. Una manera simple de conseguir pulsaciones del
teclado es utilizar la funcion getch(). El modo cbreak debe estar habilitado
para leer las teclas individualmente en lugar de lineas completas de texto (que
por lo general terminan con un retorno de carro). Debe habilitar keypad para
obtener las teclas de funcion, flecha, etc. Vea la sección Inicialización para
más detalles.\\

getch() devuelve un entero correspondiente a la tecla pulsada. Si es un
caracter normal, el valor entero será equivalente a el caracter. De lo
contrario, devuelve un número que pueda ser igualado con las constantes
definidas en curses.h. Por ejemplo, si el usuario presiona F1, el entero
devuelto es 265. Esto se puede comprobar con la macro KEY\_F() defina en
curses.h. Esto vuelve la lectura portáble y fácil de manejar.

Por ejemplo, si llama a getch() como

\begin{minted}{cpp}
  int ch;
  ch = getch();
\end{minted}

getch() esperará a que el usuario presione una tecla, (a menos que especique un
tiempo de espera) y cuando el usuario pulsa una tecla, se devuelve el entero
correspondiente . A continuación, puede comprobar el valor correspondiente
devuelto con las constantes definidas en curses.h contra las teclas que
deseadas.\\

El siguiente fragmento de código hara ese trabajo.

\begin{minted}{cpp}
  if(ch == KEY_LEFT)
    printw("Left arrow is pressed\n");
\end{minted}

Vamos a escribir un pequeño programa que crea un menú navegable con las flechas
arriba y abajo.

\subsection{Ejemplo simple de uso de teclado}%
\href{https://github.com/nasciiboy/NCURSES-Programming-HOWTO/blob/master/ncurses_programs/basics/simple_key.c}{*Ejemplo 10. Uso sencillo del teclado*}
\inputminted{cpp}{./cpp/010_uso_teclado.cpp}

\section{Interfaz con el ratón}%
Ahora hemos visto cómo obtener informacion del teclado, hagamos los mismo desde
el ratón. Por lo general, cada interfaz de usuario permite al usuario
interactuar con ambos, teclado y ratón.
\subsection{Lo basico}%
Antes de hacer cualquier otra cosa, los eventos que desea recibir deben ser
habilitados con mousemask().

\begin{minted}{cpp}
  mousemask( mmask_t newmask,     /* Los eventos que quiere escuchar */
             mmask_t *oldmask );  /* La anterior mascara de eventos */
\end{minted}

El primer parámetro de la función anterior es una máscara de bits de los
eventos que le gustaría escuchar. Por defecto, todos los eventos están
apagados. La máscara de bits ALL\_MOUSE\_EVENTS se puede utilizar para obtener
todos los eventos.\\

Las siguientes son todas las máscaras de eventos:

\begin{table}[htbp]
  \begin{center}
    \begin{tabular}{|l|l|}
      \hline
      Name & Descripcion \\
      \hline \hline
      BUTTON1\_PRESSED & boton 1 del mouse abajo \\ \hline
      BUTTON1\_RELEASED & boton 1 del mouse arriba \\ \hline
      BUTTON1\_CLICKED & boton 1 del mouse clic \\ \hline
      BUTTON1\_DOUBLE\_CLICKED &  boton 1 del mouse doble clic \\ \hline
      BUTTON1\_TRIPLE\_CLICKED &  boton 1 del mouse triple clic \\ \hline
      BUTTON2\_PRESSED & boton 2 del mouse abajo \\ \hline
      BUTTON2\_RELEASED &  boton 2 del mouse arriba \\ \hline
      BUTTON2\_CLICKED & boton 2 del mouse clic \\ \hline
      BUTTON2\_DOUBLE\_CLICKED &  boton 2 del mouse doble clic \\ \hline
      BUTTON2\_TRIPLE\_CLICKED &  boton 2 del mouse triple clic \\ \hline
      BUTTON3\_PRESSED & boton 3 del mouse abajo \\ \hline
      BUTTON3\_RELEASED &  boton 3 del mouse arriba \\ \hline
      BUTTON3\_CLICKED & boton 3 del mouse clic \\ \hline
      BUTTON3\_DOUBLE\_CLICKED &  boton 3 del mouse doble clic \\ \hline
      BUTTON3\_TRIPLE\_CLICKED &  boton 3 del mouse triple clic \\ \hline
      BUTTON4\_PRESSED & boton 4 del mouse abajo \\ \hline
      BUTTON4\_RELEASED &  boton 4 del mouse up \\ \hline
      BUTTON4\_CLICKED & boton 4 del mouse clic \\ \hline
      BUTTON4\_DOUBLE\_CLICKED &  boton 4 del mouse doble clic \\ \hline
      BUTTON4\_TRIPLE\_CLICKED &  boton 4 del mouse triple clic \\ \hline
      BUTTON\_SHIFT &  se pulso shift durante el cambio de estado del boton \\ \hline
      BUTTON\_CTRL & se pulso control durante el cambio de estado del boton \\ \hline
      BUTTON\_ALT &  se pulso alt durante el cambio de estado del boton \\ \hline
      ALL\_MOUSE\_EVENTS &  informa todos los cambios de estados \\ \hline
      REPORT\_MOUSE\_POSITION & informa movimiento del mouse \\ \hline
    \end{tabular}
  \end{center}
\end{table}

\subsection{Obtención de eventos}%
na vez que una clase de eventos de ratón se ha habilitado, las funciones de
clase getch() devuelven KEY\_MOUSE cada vez que sucede un evento de ratón. En
ese caso, el evento de ratón puede ser recuperado con getmouse().\\

El código es aproximadamente así:

\begin{minted}{cpp}
  MEVENT even;
  ch = getch();
  if(ch == KEY_MOUSE)
    if(getmouse(&event) == OK)
      /* Hacer algo con el evento */
      /* ... */
\end{minted}

getmouse() devuelve el evento en el puntero que se le da. Es un estructura que contiene

\begin{minted}{cpp}
  typedef struct
  {
    short id;         /* ID para distinguir multiples dispositivos */
    int x, y, z;      /* coordenadas del evento */
    mmask_t bstate;   /* bits de estado del boton */
  }
\end{minted}

bstate es la variable que mayor nos interesa. Informa el estado del botón del ratón.\\

Luego, con un fragmento de código como el siguiente, podemos averiguar lo sucedido.

\begin{minted}{cpp}
  if(event.bstate & BUTTON1_PRESSED)
    printw("Boton izquierdo presionado");
\end{minted}

\subsection{Poniendo todo junto}%
Esta es más o menos la interfaz con el ratón. Vamos a crear el mismo menú y
permitir la interacción del ratón. Para hacer las cosas más simples, se elimina
el manejo del teclado.\\

\href{https://github.com/nasciiboy/NCURSES-Programming-HOWTO/blob/master/ncurses_programs/basics/mouse_menu.c}{*Ejemplo 11. Acceder al menú con el ratón!!*} 
\inputminted[fontsize=\footnotesize]{cpp}{./cpp/011_menu_raton.cpp}

\subsection{Funciones varias}%
Puede utilizar las funciones mouse\_trafo() y wmouse\_trafo() para convertir las
cordenadas del ratón en coordenadas relativas a la pantalla. Vea la pagina de
manual =curs\_mouse(3X)= para más detalles.\\

La función mouseinterval establece el tiempo máximo (en milesimas de segundo)
que puede transcurrir entre eventos de presion y liberación para que puedan ser
reconocidos como un clic. Esta función devuelve el valor del intervalo
anterior. El valor predeterminado es una quinta parte de un segundo.

\section{Manipulación de Pantalla}%
En esta sección, veremos algunas funciones, que nos permiten manejar la
pantalla de manera eficiente para escribir algunos programas fantasticos. Esto
es especialmente importante en la escritura de juegos.

\subsection{Funciones getyx()}%
La función getyx() se utiliza para descubrir las coordenadas actuales del
cursor. Se colocaran las coordenadas de los valores x e y en los argumentos
dados. Ya que getyx() es una macro no tiene que pasar la dirección de las
variables. Se le puede llamar como

\begin{minted}{cpp}
  getyx(win, y, x);
  /* win : puentero a ventana
   * y, x: se colocaran las coordenadas y, x en estas variables
   */
\end{minted}

La función getparyx() obtiene las coordenadas iniciales de la sub ventana con
respecto a la ventana principal. Esto algunas veces es útil para actualizar una
sub ventana. Cuando se diseñan cosas asombrosas como menús múltiples, debido a
la dificultad de almacenar las posiciones del menú, las coordenadas de la
opcion principal etc. Una solución simple a este problema, es la creación de
menús en ventanas secundarias y luego localizar las coordenadas iniciales de
los menús utilizando getparyx().\\

Las funciones getbegyx() y getmaxyx() almacenan el inicio y las coordenadas
maximas de la ventana actual. Estas funciones son útiles de la misma manera que
la anterior en el manejo eficaz de ventanas y sub ventanas.

\subsection{Volcado de Pantalla}%
Cuando se escriben juegos, en ocaciones es necesario almacenar el estado de la
pantalla y restaurarlo de nuevo al mismo estado. Puede utilizar la función
scr\_dump() para volcar los contenidos de la pantalla al archivo dado como
argumento. Más tarde se puede restaurare con la función scr\_restore. Con estas
dos simples funciones pueden mantener el manejo eficiente en un juego de
movimiento rapido con cambios de escenarios.

\subsection{Volcado de Ventana}%
Puede utilizar las funciones putwin() y getwin(), para guardar y restaurar
ventanas. putwin() coloca el estado actual de la ventana en un archivo, que
puede ser restaurado posteriormente por getwin().\\

Puede utilizar la función copywin() para copiar por completo una ventana en
otra ventana. Toma las ventanas de origen y destino como parámetros y de
acuerdo con el rectángulo especificado, copia la región rectangular desde la
ventana origen hasta la ventana destino. El último parámetro especifica si
desea sobrescribir o simplemente superponer el contenido en la ventana destino.
Si este argumento es cierto, entonces la copia es no destructiva.

\section{Características diversas}%
Ahora ya conoces suficientes características para escribir un buen programa
curses. Aqui estan algunas funciones auxiliares útiles en varios casos. Vamos a
ir de cabeza a algunas de ellas.

\subsection{curs\_set()}%
Puede utilizar esta función para hacer el cursor invisible. Los parámetros
aceptados por esta funcion son.
\begin{itemize}
  \item 0 : invisible
  \item 1 : normal
  \item 2 : muy visible.
\end{itemize}

\subsection{Dejar Temporalmente el modo curses}%
En algunas ocaciones puede buscar volver temporalmente al cooked mode(modo
normal de linea de buffer). En tal caso, primero tendrás que guardar los modos
tty con una llamada a def\_prog\_mode() y luego llamar a endwin() para
finalizar el modo de curses. Esto te dejará en el el modo tty original. Una vez
que haya terminado, llamara a reset\_prog\_mode() para volver al modo curses.
Esta función devuelve el tty al estado almacenada por def\_prog\_mode(). Luego
de un refreth(), estára de vuelta en el modo de curses. He aquí un ejemplo que
muestra la secuencia de cosas por hacer.

\href{https://github.com/nasciiboy/NCURSES-Programming-HOWTO/blob/master/ncurses_programs/basics/temp_leave.c}{*Ejemplo 12: Dejando temporalmente el Modo curses}
\inputminted{cpp}{./cpp/012_dejando_temp_curses.cpp}

\subsection{Variables ACS\_}%
Si alguna vez has programado en DOS, debes conocer acerca de este ingenioso
juego de caracteres extendidos. Estos son imprimibles sólo en algunos
terminales. Las funciones como box() utilizan estos caracteres. Todas estas
variables comienzan con ACS significando el conjunto de caracteres alternativo.
Es posible que haya observado mi uso de estos caracteres en algunos de los
programas anteriores. El siguiente ejemplo que muestra todos los caracteres.

\href{https://github.com/nasciiboy/NCURSES-Programming-HOWTO/blob/master/ncurses_programs/basics/acs_vars.c}{*Ejemplo 13: Variables ACS*}
\inputminted{cpp}{./cpp/013_variables_ACS.cpp}

\section{Otras librerias}%
Además de la libreria curses, hay algunas librerias en modo texto, que
proporcionan una mayor funcionalidad y un montón de características. Las
siguientes secciones describen tres librerias estándar que normalmente se
distribullen junto a curses.

\section{Libreria Panel}%
Ahora que eres competente en curses, buscaras hacer algo grande. Como crear una
gran cantidad de ventanas superpuestas para dar un aspecto profesional tipo
ventana. Por desgracia, esto se hace dificil de manejar en poco tiempo. Las
actualizaciones múltiples, te sumergiran en una pesadilla. Las ventanas
superpuestas generan manchas, cada vez que olvida actualizar las ventanas en el
orden correcto.\\

No caiga en la desesperacion. Hay una solución elegante proporcionada en la
libreria Parel. En palabras de los desarrolladores de ncurses:\\

Si tiene muchas ventanas superpuestas, la libreria panel es el camino a seguir.
Es obvia la necesidad de realizar una serie de wnoutrefresh(), doupdate() y
aliviar la carga de hacerlo correctamente (abajo arriba). La libreria mantiene
información sobre el orden de las ventanas, su superposición y en consecuencia
actualiza la pantalla correctamente. Así que ¿por qué esperar? Echemos un
vistazo de cerca a los paneles.

\subsection{Lo basico}%
Un objeto panel es una ventana que se trata implícitamente como parte de una
baraja incluyendo a todos los otros objetos panel. La baraja se trata como una
pila con un panel superior que es completamente visible y los otros paneles
pueden o no ser oscurecidos respecto a sus posiciones. Asi que La idea básica
es crear una pila de paneles superpuestos y utilizal la libreria panel para
desplegarlos correctamente. Hay una función similar a refresh() que, cuando es
llamada, despliega los paneles en el orden correcto. Se proporcionan funciones
para ocultar o mostrar paneles, mover paneles, cambiar su tamaño, etc.. El
problema de superposicion es gestionado por la libreria panel durante todas las
llamadas a estas funciones.\\

El flujo general de un programa panel es el siguiente:
\begin{itemize}
  \item Crear las ventanas (con newwin()) que seran ligadas a paneles.
  \item Cree paneles con el ordend de visibilidad elegido. Apilarlos de acuerdo
    con la visibilidad deseada. La función new\_panel() se utiliza para crear
    paneles.
  \item Llame a update\_panels() para escribir los paneles en la pantalla
    virtual en el orden correcto de visibilidad. llame a doupdate() para
    mostrarlo en pantalla.
  \item Manipule los paneles con show\_panel(), hide\_panel(), move\_panel() etc.
    Haga uso de funciones auxiliares como panel\_hidden() y panel\_window(). Haga
    uso del puntero de usuario para almacenar datos personalizados para un
    panel. Utilice las funciones set\_panel\_userptr() y panel\_userptr() para
    establecer y obtener el puntero de usuario de un panel.
  \item Cuando haya terminado de utilizar el panel llame a del\_panel() para
    eliminar el panel.
\end{itemize}

Hagamos los conceptos claros, con algunos programas. El siguiente es un
sencillo programa que crea 3 paneles superpuestos y les muestra en la pantalla.

\subsection{Compilación con la Librería Panel}%
Para utilizar las funciones de la libreria panel, tiene que incluir panel.h y
vincular el programa a la dicha libreria con la bandera -lpanel junto con
-lncurses en ese orden.

\begin{minted}{cpp}
  #include <panel.h>
  .
  .
  .
\end{minted}

Compilación y enlace:
\mint{bash}|gcc <fichero del programa> -lpanel -lncurses|

\href{https://github.com/nasciiboy/NCURSES-Programming-HOWTO/blob/master/ncurses_programs/panels/panel_simple.c}{*Ejemplo 14: Panel conceptos basicos*}
\inputminted{cpp}{./cpp/014_panel.cpp}

Como se puede ver, el programa anterior sigue un flujo simple. Las ventanas se
crean con newwin() y luego son unidas a paneles con new\_panel(). La pila de
paneles se actualiza, a medida que agregamos un panel después otro. Para
mostrarlos en pantalla llamamos a update\_panels() y doupdate().

\subsection{Navegacion en Paneles de Ventana}%
El siguiente es un ejemplo un poco complicado. Este programa crea 3 ventanas
que se puede rotar ciclicamente mediante el uso de tab. Echa un vistazo al
código.

\href{https://github.com/nasciiboy/NCURSES-Programming-HOWTO/blob/master/ncurses_programs/panels/panel_browse.c}{*Ejemplo 15: Navegación en Paneles de Ventana}
\inputminted{cpp}{./cpp/015_navegacion_paneles.cpp}

\subsection{Utilizando punteros de usuario}%
En el ejemplo anterior utilice punteros de usuario para obtener la siguiente
ventana en el ciclo. Podemos vincular información personalizada al panel
especificando un puntero de usuario, que puede apuntar a cualquier información
desea almacenar. En este caso guardé el puntero al siguiente panel en el ciclo.
El puntero de usuario de un panel se puede establecer con la función
set\_panel\_userptr(). Se puede acceder a dicha informacion mediante la función
panel\_userptr() que devolverá el puntero de usuario de el panel dado como
argumento. Después localizar el siguiente panel en el ciclo se establece en la
cima con la función top\_panel(). Esta función trae a el panel dado como
argumento a la parte superior de la pila de paneles.

\subsection{Mover y Redimensionar Paneles}%
Puede utilizar la función move\_panel() para mover un panel a la posicion
deseada. Esto no cambia la posición del panel en la pila. Asegúrese utilizar
move\_panel() en lugar de mvwin() en la ventana asociada a el panel.\\

Cambiar el tamaño de un panel es ligeramente complejo. No hay una función
directa para cambiar el tamaño de la ventana asociada a un un panel. Una
solución para cambiar el tamaño de un panel es crear una nueva ventana con el
tamaño deseado, cambie la ventana asociada a el panel utilizando
replace\_panel(). No olvides borrar la ventana anterior. Puede obtener la
ventana asociada con un panel mediante el uso de la función panel\_window().\\

El siguiente programa muestra estos conceptos, en suposicion es un programa
sencillo. Como es costumbre, puede rotar entre ventanas con <TAB>. Para cambiar
el tamaño o mover el panel activo presione “r” por “resize”, ‘m’ para mover. A
continuación, utilice las teclas de flecha para cambiar el tamaño o mover a la
forma deseada y pulse enter para terminar el cambio. En ejemplo hace uso de los
datos del usuario para obtener los datos necesarios para hacer las operaciones.

\href{https://github.com/nasciiboy/NCURSES-Programming-HOWTO/blob/master/ncurses_programs/panels/panel_resize.c}{*Ejemplo 16: Mover y redimencionar un Panel}
\inputminted{cpp}{./cpp/016_mover_redimencionar_panel.cpp}

Concéntrese en el bucle while principal. Una vez que se descubre que tecla se
ha pulsado, se toma la accion correspondiente. Si se presiona ‘r’ se inicia el
modo de tamaño. Después de esto, los nuevos tamaños se actualizan a medida que
el usuario presiona las teclas de flecha. Cuando el usuario pulsa <ENTER>
termina la presente selección y el panel cambia de tamaño utilizando el
concepto explicado anteriormente. Dentro del modo de tamaño el programa no
muestra cómo se redimensiona la ventana. Se deja como ejercicio para el lector
imprimir un borde de puntos mientras se cambia el tamaño a una nueva posición.\\

Cuando el usuario presiona ‘m’ inicia el modo de movimiento. Esto es un poco
más simple que el cambio de tamaño. Cuando pulsa las teclas de flecha se
actualiza la nueva posición y al presionar <ENTER> el panel es movido llamando
a la función move\_panel().\\

En este programa los datos de usuario representados como PANEL\_DATA, juega
papel muy importante en la búsqueda de la información asociada a un panel. Como
está escrito en los comentarios, PANEL\_DATA almacena el tamaño del panel, la
etiqueta, color de la etiqueta y un puntero al siguiente panel en el ciclo.

\subsection{Ocultar y Mostrar Paneles}%
Un panel se puede ocultar mediante el uso de la función hide\_panel(). Esta
función simplemente lo quita de la pila de paneles, de esta forma lo oculta en
pantalla cuando llama a update\_panels() y doupdate(). Esto no destruye la
estructura PANEL asociada con el panel oculto. Puede mostrar nuevamente el
panel utilizando la funcion show\_panel().\\

El siguiente programa muestra como ocultar paneles. Pulse ‘a’ o ‘b’ o ‘c’ para
mostrar u ocultar la primera, segunda y tercera ventana respectivamente. Se
utilizan los datos de usuario con la pequeña variable hide, que realiza un
seguimiento de si la ventana está oculta o no. Por alguna razón la función
panel\_hidden() que indica si un panel está oculto o no, no está funcionando. Un
informe de error fue también presentado por Michael Andres aqui

\href{https://github.com/nasciiboy/NCURSES-Programming-HOWTO/blob/master/ncurses_programs/panels/panel_hide.c}{*Ejemplo 17: Ocultar y mostrar Paneles}
\inputminted{cpp}{./cpp/017_ocultar_mostrar_paneles.cpp}

\subsection{Funciones panel\_above() y panel\_below()}%
Puede utilizar las funciones panel\_above() y panel\_below() para averiguar el
panel superiar e inferiar a un panel. Si el argumento de estas funciones es
NULL, devuelven un puntero al panel inferior y superior, respectivamente.

\section{Libreria Menu}%
La libreria menu proporciona una buena extensión al curses básico, a través de
esta se pueden crear menús. Proporciona un conjunto de funciones para crear
menús. Pero tiene que personalizarce para dar un aspecto más agradable, con
colores etc. Entremos a los detalles.\\

Un menú es una visualización de pantalla que ayuda al usuario a elegir un
cierto subconjunto de un conjunto dado de elementos. Para ponerlo simple, un
menú es una colección de elementos de los que se pueden elegir uno o más
elementos. Algunos lectores podrían no ser conscientes de la capacidad de
seliccion de elementos multiples. La libreria menu ofrece funcionalidad para
escribir menús de los cuales es usuario puede elegir más de un elemento. Esto
se trata en una sección posterior. Ahora es el momento para algunas nociones.

\subsection{Lo basico}%
Para crear menús, primero debe crear elementos, y luego colocar el menú en la
pantalla. Después de eso, todo el procesamiento de las respuestas del usuario
es hecho en una función elegante llamada menu\_driver() que es el caballo de
trabajo de cualquier programa de menú.\\

El flujo general de control de un programa de menú se parece a esto.

\begin{itemize}
  \item Inicializar curses
  \item Cree elementos usando new\_item(). Puede especificar un nombre y una
    descripción de los elementos.
  \item Crear el menú con new\_menu() especificando los elementos que se
    agregaran a este.
  \item Publicar el menú con menu\_post() y refrescar la pantalla.
  \item Procesar las las peticiones de los usuarios con un bucle y hacer las
    actualizaciones necesarias al menú con menu\_driver.
  \item Quitar el menú con menu\_unpost()
  \item Liberar la memoria asignada al menú con free\_menu()
  \item Liberar la memoria asignada a los elementos con free\_item()
  \item Finalizar curses
\end{itemize}

Vamos a ver un programa que imprime un menú sencillo y actualiza la selección
actual con las teclas de flecha arriba, abajo.

\subsection{Compilación con la Libreria Menú}%
Para utilizar las funciones de libreria de menu, tiene que incluir menu.h y
enlazar el programa con la libreria menu mediante la bandera -lmenu debe ser
añadido junto con -lncurses en ese orden.

\begin{minted}{cpp}
  #include <menu.h>
  .
  .
  .
\end{minted}

Compilación y enlace:

\mint{bash}|gcc <fichero del programa> -lmenu -lncurses|

\href{https://github.com/nasciiboy/NCURSES-Programming-HOWTO/blob/master/ncurses_programs/menus/menu_simple.c}{*Ejemplo 18: Conceptos básicos de Menu*}
\inputminted{cpp}{./cpp/018_conceptos_menu.cpp}

Este programa demuestra los conceptos básicos que intervienen en la creación de
un menú con libreria menu. Primero creamos los elementos mediante new\_item() y
luego los agregamos al menu con la funcion new\_menu(). Después de publicar el
menu y actualizar la pantalla, inicia el bucle principal de procesamiento. Se
lee la entrada del usuario y se toma la acción correspondiente. La función
menu\_driver() es el principal caballo de trabajo del sistema de menus. El
segundo parámetro para esta función dice lo que hay que hacer con el menu. De
acuerdo con el parámetro, menu\_driver() hace la tarea correspondiente. El valor
puede ser ya sea una solicitud de navegación, un carácter ASCII, o una clave
especial KEY\_MOUSE asociada a un evento de ratón.\\

menu\_driver acepta las siguientes solicitudes de navegación.

\begin{itemize}
  \item REQ\_LEFT\_ITEM        Mover un elemento a la izquierda.
  \item REQ\_RIGHT\_ITEM       Mover un elemento a la derecha.
  \item REQ\_UP\_ITEM          Mover un elemento hacia arriba.
  \item REQ\_DOWN\_ITEM        Mover un elemento hacia abajo.
  \item REQ\_SCR\_ULINE        Desplácese una línea arriba.
  \item REQ\_SCR\_DLINE        Desplácese una línea abajo.
  \item REQ\_SCR\_DPAGE        Desplazarse una página hacia arriba.
  \item REQ\_SCR\_UPAGE        Desplazarse una página hacia abajo.
  \item REQ\_FIRST\_ITEM       Mover al primer elemento.
  \item REQ\_LAST\_ITEM        Mover al último elemento.
  \item REQ\_NEXT\_ITEM        Mover al siguiente elemento.
  \item REQ\_PREV\_ITEM        Mover al elemento anterior.
  \item REQ\_TOGGLE\_ITEM      Seleccionar/deseleccionar un elemento.
  \item REQ\_CLEAR\_PATTERN    Borrar el buffer de patrones menu.
  \item REQ\_BACK\_PATTERN     Eliminar el carácter anterior desde el buffer de patrones.
  \item REQ\_NEXT\_MATCH       Mover al elemento siguiente que coincida con el patron.
  \item REQ\_PREV\_MATCH       Mover al elemento anterior que coincida con el patron.
\end{itemize}

No se deje abrumar por el número de opciones. Vamos a verlos lentamente una
tras otro. Las opciones de interés en este ejemplo son REQ\_UP\_ITEM y
REQ\_DOWN\_ITEM. Estas dos opciones son pasadas a menu\_driver, que actualiza el
elemento actual a un elemento hacia arriba o hacia abajo, respectivamente.

\subsection{Menu Driver: El caballo de trabajo del sistema de menus}%
Como se ha visto en el ejemplo anterior, menu\_driver juega un papel importante
en la actualización del menu. Es muy importante comprender las diversas
opciones y lo que hacen. Como se ha explicado, el segundo parámetro de
menu\_driver() puede ser una solicitud de navegación, un carácter imprimible o
una clave KEY\_MOUSE. Vamos a analizar las diferentes solicitudes de navegación.

\textbf{REQ\_LEFT\_ITEM y REQ\_RIGHT\_ITEM}\\
Se puede visualizar un menu con varias columnas de más de un ítem. Esto se
realiza mediante la funcion menu\_format(). Cuando se muestra un menu de
columnas múltiples estas peticiones hacen que menu\_driver de menu mueva la
eleccion actual a la izquierda o la derecha.\\

\textbf{REQ\_UP\_ITEM and REQ\_DOWN\_ITEM}\\
Estas dos opciones se han visto en el ejemplo anterior. Cuando se pasan estas
opciones, hacen que menu\_driver mueva la eleccion actual un elemento hacia
arriba o hacia abajo.\\

\textbf{Opciones REQ\_SCR\_*}\\
Las cuatro opciones REQ\_SCR\_ULINE, REQ\_SCR\_DLINE, REQ\_SCR\_DPAGE,
REQ\_SCR\_UPAGE están relacionadas con el desplazamiento. Si no se pueden
mostrar todos los elementos de el menu en la subventana menu, el menu es
desplazable. Estas solicitudes se pueden pasar a menu\_driver para realizar el
desplazamiento, ya sea por línea o por pagina hacia arriba o hacia abajo
respectivamente.\\

\textbf{REQ\_FIRST\_ITEM, REQ\_LAST\_ITEM, REQ\_NEXT\_ITEM and REQ\_PREV\_ITEM}\\
Estas solicitudes se explica por sí mismas.\\

\textbf{REQ\_TOGGLE\_ITEM}\\
Quando se pasa esta solicitud, se alterna la selección actual. Solo se utiliza
esta opción en un menu de valores múltiples. Para utilizar esta solicitud la
opción O\_ONEVALUE debe estar apagada. Esta opción puede ser apagar o encender
con set\_menu\_opts().\\

\textbf{Solicitud por Patrones}\\
Cada menu tiene una buffer de patrones asociado, que se utiliza para encontrar
la coincidencia más cercana a los caracteres ASCII introducidos por el usuario.
Siempre que se pasan caracteres ASCII a menu\_driver, se colocan el el buffer de
patrones. Este inteta encontrar la coincida más cercana con el patrón en la
lista de elementos y mueve la seleccion a ese elemento. La solicitud
REQ\_CLEAR\_PATTERN limpia el buffer de patrones. La solicitud REQ\_BACK\_PATTERN
borra el carácter anterior en el buffer de patrones. En caso de que el patrón
coincida con mas de un elemento se puede rotar por los elementos coincidentes
con REQ\_NEXT\_MATCH y REQ\_PREV\_MATCH que mueven la selección a la conicidencia
siguiente y anterior, respectivamente.\\

\textbf{Solicitudes del raton}\\
En caso de solicitudes KEY\_MOUSE, de acuerdo con la posición del ratón se toma
una acción en consecuencia. La acción a tomar se explica en la página man como,\\
Si el segundo argumento es la clave especial KEY\_MOUSE, el evento de ratón
asociado se traduce en una de las anteriores solicitudes pre-definidas.
Actualmente, solo se manejan los clics en la ventana de usuario (por ejemplo,
dentro del área de visualización del menu o la decoracion de ventana). Si hace
clic sobre la zona de visualizacion del menu, se genera un REQ\_SCR\_ULINE, con
doble clic se genera REQ\_SCR\_UPAGE y con triple clic REQ\_FIRST\_ITEM. Si hace
clic por debajo de la zona de visualización del menu, se genera un
REQ\_SCR\_DLINE, un doble clic REQ\_SCR\_DPAGE y un triple click genera
REQ\_LAST\_ITEM. Si hace clic en un elemento dentro del área de visualización del
menu, el cursor del menu se coloca en dicho elemento.\\

Cada una de las solicitudes anteriores se explicará en las siguientes líneas
con varios ejemplos siempre que sea apropiado.

\subsection{17.4. Ventanas de Menu}%
Cada menu creado se asocia con una ventana y una sub ventana. La ventana menu
muestra cualquier título o borde asociado con el menu. La ventana submenu
muestra los elementos del menu actualmente disponible para su selección. Pero
no especificamos ninguna ventana o subventana en el ejemplo. Cuando no se
especifica una ventana, stdscr se toma como la ventana principal y, el sistema
de menus calcula el tamaño requerido de la subventana para la visualización de
elementos. A continuación se muestran los elementos en la subventana. Asi que
vamos a jugar con estas ventanas, mostrado un menu con un borde y un título.\\

\href{https://github.com/nasciiboy/NCURSES-Programming-HOWTO/blob/master/ncurses_programs/menus/menu_win.c}{*Ejemplo 19. Uso de Ventanas Menu}
\inputminted{cpp}{./cpp/019_uso_ventanas_menu.cpp}

En este ejemplo se crea un menu con un título, margen, una fantastica línea
separa el título y los elementos. Como puede ver, con el fin de unir una
ventana a un menu se utiliza la función set\_menu\_win(). A continuación
agregamos tambien la ventana secundaria. Esto despliega los elementos en la
subventana. También puede establecer la cadena de marca que se muestra a la
izquierda del elemento seleccionado con set\_menu\_mark().

\subsection{Desplazamiento en Menus}%
Si la subventana dada no es lo suficientemente grande para mostrar todo los
elementos, el menu serán desplazable. Cuando se está en el último elemento de
la lista actual, si envía REQ\_DOWN\_ITEM, sera traducido como REQ\_SCR\_DLINE y el
menu se desplaza un elemento. Puede pasar manualmente operaciones REQ\_SCR\_ para
realizar el desplazamiento. Veamos cómo se puede realizar.\\

\href{https://github.com/nasciiboy/NCURSES-Programming-HOWTO/blob/master/ncurses_programs/menus/menu_scroll.c}{*Ejemplo 20. Menu con desplazamiento}
\inputminted{cpp}{./cpp/020_desplazamiento_menu.cpp}

Este programa se explica por sí. En este ejemplo, el número de opciones ha sido
aumentado a diez, que es más grande que el tamaño que nuestra subventana puede
albergar (6 elementos). Este mensaje tiene que ser transmitido explícitamente
al sistema de menus con la función set\_menu\_format(). Aquí especificamos el
número de filas y columnas que queremos que se muestren para una sola página.
Podemos especificar cualquier número de elementos a mostrar, en la variable
rows, si este es menor que la altura de la subventana. Si la tecla presionada
por el usuario es PAGE UP o PAGE DOWN, el menu se desplaza una página debido a
la las solicitudes (REQ\_SCR\_DPAGE y REQ\_SCR\_UPAGE) dadas a menu\_driver().

\subsection{Menu de Multiples Columnas}%
En el ejemplo anterior se vio cómo utilizar la función set\_menu\_format(). No
mencioné lo que hace la variable cols (tercer parámetro). Si la subventana es
lo suficientemente amplia, puede optar por mostrar más de un elemento por fila.
Esto se puede especificar en la variable cols. Para aclarar las cosas, El
siguiente ejemplo no muestra la descripcion de los elementos.\\

\href{https://github.com/nasciiboy/NCURSES-Programming-HOWTO/blob/master/ncurses_programs/menus/menu_multi_column.c}{*Ejemplo 21. Menu con Columnas Multiples}
\inputminted{cpp}{./cpp/021_menu_columnas.cpp}

Ver la llamada a la función = set\_menu\_format()=. Especifica el número de
columnas a ser 3, mostrando de este modo 3 artículos por fila. Tenemos También
apagado las descripciones que muestran con la función menu\_opts\_off()=. Hay par
de funciones = set\_menu\_opts()=, menu\_opts\_on()= y = menu\_opts()= que se pueden
utilizar para manipular opciones de menu. Las siguientes opciones de menu se
pueden especificar.\\

\textbf{O\_ONEVALUE}\\
Solo un elemento puede seleccionarse para este menu.\\

\textbf{O\_SHOWDESC}\\
Mostrar las descripciones de los elementos cuando el menu es publicado.\\

\textbf{O\_ROWMAJOR}\\
Mostrar el menu en orden de importancia de las filas.\\

\textbf{O\_IGNORECASE}\\
Ignorar el caso en cuando el patron coincida.\\

\textbf{O\_SHOWMATCH}\\
Mueve el cursor dentro del nombre del elemento, mientras el patron coincide.\\

\textbf{O\_NONCYCLIC}
No rotar entre el siguiente/anterior elemento, solicitar por por el otro extremo del menu.\\

Todas las opciones están habilitadas de forma predeterminada. Puede cambiar los
atributos específicos a encendido o apagado con las funciones menu\_opts\_on() y
menu\_opts\_off(). También puede utilizar set\_menu\_opts() para especificar
directamente las opciones. El argumento para esta función debe ser un valor OR
de algunas de las constantes anteriores. Puede utilazar la función menu\_opts()
para optener las opciones actuales de un menu.

\subsection{Menu de Valores Multiples}%
Tal vez se pregunte lo que ocurre si apaga la opción O\_ONEVALUE. Esto crea un
menu de valores multiples. Eso significa que se puede seleccionar más de un
elemento. Esto nos lleva a la solicitud REQ\_TOGGLE\_ITEM. Veamoslo en acción.\\

\href{https://github.com/nasciiboy/NCURSES-Programming-HOWTO/blob/master/ncurses_programs/menus/menu_toggle.c}{*Ejemplo 22. Menú de valores Múltiples}
\inputminted{cpp}{./cpp/022_menu_valores_multiples.cpp}

Menos mal, un montón de nuevas funciones. Vamos a tomar una tras otra. En
primer lugar, REQ\_TOGGLE\_ITEM. En un menu de valores multiples, se debe
permitir a el usuario seleccionar o anular más de un elemento. La solicitud
REQ\_TOGGLE\_ITEM alterna la selección actual. En esto caso cuando se pulsa
espacio se envia la solicitud REQ\_TOGGLE\_ITEM a menu\_driver para conseguir el
resultado.\\

Ahora, cuando el usuario presiona <ENTER> se muestran los elementos actualmente
seleccionados. Primero debemos encontrar los elementos asociados con el menu
empleando la función menu\_items(). A continuación, recorrer los elementos para
saber si el elemento se selecciona o no. La función item\_value() devuelve TRUE
si el elemento se ha seleccionado. La función item\_count() devuelve el número
de elementos en el menu. El objeto nombre se puede optener con item\_name().
También puede optener la descripción asociado con un elemento utilizando
item\_description().

\subsection{Opciones de Menu}%
Bueno, a estas alturas ya debe estar buscando alguna diferencia en su menu, con
un montón de funcionalidades. Lo sé. Quiere Colores !!!. Quiere crear menus
agradables similares a las de aquellos juegos dos en modo texto. Puede utilizar
las funciones set\_menu\_fore() y set\_menu\_back() para cambiar el atributo de los
elementos seleccionados y no seleccionados. Los nombres son engañosos. No
cambian el primer plano o el fondo del menu lo cual los haria menos utiles.\\

Puede utilizar la función set\_menu\_grey() para establecer el atributo de
pantalla para los elementos no seleccionables en el menu. Esto nos lleva a la
interesante opción para un unico elemento O\_SELECTABLE. Podemos apagar dicha
opcion con la función item\_opts\_off() y luego el elemento deja de ser
seleccionable. Esto es parecido a atenuar un elemento en alquellos fantasticos
menus. Vamos a poner estos conceptos en práctica con este ejemplo\\

\href{https://github.com/nasciiboy/NCURSES-Programming-HOWTO/blob/master/ncurses_programs/menus/menu_attrib.c}{*Ejemplo 23. Opciones del Menú*}
\inputminted{cpp}{./cpp/023_opciones_menu.cpp}

\subsection{El conveniente Puntero de Usuario}%
Podemos asociar un puntero de usuario a cada elemento del menu. Esto funciona
de la misma forma que en los paneles. Esto no es tocado por el sistema de
menus. Puede almacenar cualquier cosa en estos. Yo suelo utilizarlo para
almacenar la función a ejecutar cuando se elige la opcion de menu (Esta
seleccionada y el usuario pulsa <ENTER>);\\

\href{https://github.com/nasciiboy/NCURSES-Programming-HOWTO/blob/master/ncurses_programs/menus/menu_userptr.c}{*Ejemplo 24. Menu con Puntero de Usuario} 
\inputminted{cpp}{./cpp/024_puntero_usuario_menu.cpp}

\section{Libreria Form}%
Bien. Si usted ha visto esos formularios en páginas web que toman la entrada
del usuario y hacen varios tipos de cosas, puede que se pregunte cómo crear
formularios en pantalla en modo texto. Es bastante difícil escribir esos
elegantes formularios en ncurses puro. La libreria form intenta proporcionar un
marco de trabajo básico para construir y mantener formularios con facilidad.
Cuenta con gran cantidad de características (funciones) que gestionan la
validación, expansión dinámica de campos, etc.. Veamos esto en pleno
desarrollo.\\

Un formulario es una colección de campos; Cada campo puede ser o bien una
etiqueta (texto estático) o un lugar de entrada de datos. La libreria también
proporciona funciones para dividir los formularios en varias páginas.

\subsection{Lo basico}%
Los formularios se crean en la misma manera que los menus. Primero se crean los
campos relacionados con el formulario con new\_field(). Puede configurar las
opciones de los campos, de manera que se pueden mostrar con ciertos atributos,
validar el contenido antes de perder el foco del campo etc.. A continuación,
los campos se agregan al formulario. Después de esto, el formulario puede ser
publicado y está listo para recibir entradas. En líneas similares a
menu\_driver(), los formularios se manejan con form\_driver(). Podemos enviar
solicitudes a form\_driver para mover el foco a un cierto campo, mover el cursor
al final del campo etc.. Después de que el usuario introduce los valores en los
campos y se termina la validación, su puede remover el formulario liberar la
menoria asignada.\\

El flujo de control general de un programa de formulario se vee como esto.\\

\begin{itemize}
  \item Inicializar curses
  \item Crear campos utilizando new\_field(). Puede especificar la altura y
    anchura del campo, y su posición en el formulario.
  \item Crear los formularios con new\_form() especificando los campos a agregar.
  \item Publicar el formulario con form\_post() y refrescar la pantalla.
  \item Procesar las peticiones de usuarios con un bucle y hacer las
    actualizaciones necesarias al formulario con form\_driver.
  \item Remover el menu con form\_unpost()
  \item Liberar la memoria asignada al menu con free\_form()
  \item Liberar la memoria asignada a los elementos con free\_field()
  \item Finalizar curses
\end{itemize}

Como puede ver, trabajar con la libreria form es muy similar a el manejo de la
libreria menu. Los siguientes ejemplos explorarán diversos aspectos de
procesamiento de formularios. Vamos a empezar el viaje con un simple ejemplo.
antes.

\subsection{Compilación con la Libreria Form}%
Para utilizar funciones de la libreria form, tiene que incluir form.h y
vincular el programa a la libreria con la bandera -lform se debe añadir junto
con -lncurses en ese orden.

\begin{minted}{cpp}
  #include <form.h>
  .
  .
  .
\end{minted}

Compilación y enlace:
\mint{bash}|gcc <fichero del programa> -lform -lncurses|

\href{https://github.com/nasciiboy/NCURSES-Programming-HOWTO/blob/master/ncurses_programs/forms/form_simple.c}{*Ejemplo 25. Formularios básicos}
\inputminted{cpp}{./cpp/025_formularios_basicos.cpp}

El ejemplo anterior es bastante sencillo. Se crean dos campos con new\_field().
new\_field() toma altura, anchura, starty, startx, número de filas fuera de la
pantalla y el número de buffers de trabajo adicional. El quinto argumento
especifica cómo mostrar gran parte del campo. Si es cero, todo el campo se
muestra simpre de otra manera el formulario será desplazable cuando el campo no
muestre partes de la entrada de usuario. La libreria de formularios asigna un
búfer por campo para almacenar los datos que el usuario ingresa. Con el ultimo
parametro de new\_field() podemos especificar si se almacena en buffers
adicionales. Estos puede utilizarse ​​para cualquier propósito.\\

Después de crear los campos, el atributo de fondo de ambos es establecido como
subrayado con set\_field\_back(). La opción AUTOSKIP se desactiva usando
field\_opts\_off(). Si esta opción está activada, el foco se desplazará al
siguiente campo en el formulario una vez que el campo activo se llene por
completo.\\

Después de colocar los campos en el formulario, se publica. Aquí, se procesa la
entrada de usuario en un bucle while, haciendo las peticiones correspondientes
a form\_driver. Los detalles de todas las solicitudes a form\_driver() se
explican más adelante.

\subsection{Jugando con Campos}%
Cada campo del formulario se asocia con una gran cantidad de atributos. Estos
se pueden manipular para conseguir el efecto deseado y divertirse !!!. Asi que
¿por que esperar?

\subsubsection{Recuperar el Tamaño y Posicion del campo}%
Los parámetros dados a el momento de la creación de un campo pueden ser
recuperados con field\_info(). Devuelve la altura, anchura, starty, startx,
número de filas fuera de la pantalla, y el número buffers adicionales en los
parámetros que le son dados. Es una especie de new\_field() inversa.

\begin{minted}{cpp}
  int field_info(     FIELD *field,              /* campo a extraer datos                */
                      int *height, *int width,   /* tamaño de campo                      */
                      int *top, int *left,       /* esquina superior izquierda           */
                      int *offscreen,            /* numero de columnas fuera de pantalla */
                      int *nbuf);                /* numero de buffers de trabajo         */
\end{minted}

\subsubsection{Mover el campo}%
Puede mover el campo a una posición diferente con move\_field().

\begin{minted}{cpp}
  int move_field(    FIELD *field,              /* campo a modificar                */
                     int top, int left);        /* nueva esquina superior izquierda */
\end{minted}

Puede consultar la posición modificada field\_infor().

\subsubsection{Justificar un campo}%
Puede justificar un campo mediante la función set\_field\_just().

\begin{minted}{cpp}
  int set_field_just(FIELD *field,       /* campo a modificar  */
                     int justmode);      /* modo de ajuste     */
  int field_just(FIELD *field);          /* obtener el modo de justificacion de un campo */
\end{minted}

Los valores aceptados y devueltos para el modo de justificación son
NO\_JUSTIFICATION, JUSTIFY\_RIGHT, JUSTIFY\_LEFT, o JUSTIFY\_CENTER.

\subsubsection{Atributos de Visualizacion del Campo}%
Como se ha visto, en el ejemplo anterior, se pueden establecer los atributos de
visualización del campo con set\_field\_fore() y set\_field\_back(). Estas
funciones establecen el atributo de campo para el primer y segundo plano.
También puede especificar un carácter de relleno que llenará la porción
restante del campo. El carácter de relleno se establece con una llamada a
set\_field\_pad(). Este valor por defecto es espacio. Puede utilizar las
funciones field\_fore(), field\_back, field\_pad() para consultar los atributos de
relleno, del primer y segundo plano. El siguiente listado muestra el uso de
estas funciones.

\begin{minted}{cpp}
  int set_field_fore(FIELD *field,        /* campo a modificar                 */
                     chtype attr);        /* atributo a establecer             */

  chtype field_fore(FIELD *field);        /* campo a consultar                 */
                                          /* devuelve atributo de primer plano */

  int set_field_back(FIELD *field,        /* campo a modificar                 */
                     chtype attr);        /* atributo a establecer             */

  chtype field_back(FIELD *field);        /* campo a consultar                 */
                                          /* devuelve atributo de fondo        */

  int set_field_pad(FIELD *field,         /* campo a modificar                */
                    int pad);             /* caracter de relleno a establecer */

  chtype field_pad(FIELD *field);         /* campo a consultar                */
                                          /* devuelve el caracter de relleno */
\end{minted}

A pesar de que el uso de las funciones anteriores parece bastante simple,
utilizar de colores con set\_field\_fore() puede ser frustrante al principio.
Primero dejame explicar acerca de los atributos de primer plano y fondo de un
campo. El atributo de primer plano está asociado con el caracter. Eso significa
que un carácter en el campo se imprime con el atributo que ha configurado con
set\_field\_fore(). El atributo de fondo es utilizado para llenar el fondo del
campo, si algún caracter está allí o no. Entonces, ¿qué hay acerca de los
colores? Los colores siempre se definen en pares, ¿cuál es la manera correcta
de mostrar colores en los campos? He aquí un ejemplo para aclara el uso de
atributos de color.\\

\href{https://github.com/nasciiboy/NCURSES-Programming-HOWTO/blob/master/ncurses_programs/forms/form_attrib.c}{*Ejemplo 26. Atributos de Campo*}
\inputminted{cpp}{./cpp/026_atributos_campo.cpp}

Juega con los pares de color y tratar de entender los atributos de de fondo y
primer plano. En mis programas cuando utilizo atributos de color, por lo
general estableco sólo el fondo con set\_field\_back(). curses simplemente no
permite definir atributos de color individuales.

\subsubsection{Bits de Opcion de Campo}%
Hay también una gran colección de bits de opcion de campo que puede configurar
para controlar varios aspectos al procesar formularios. Puede manipularlos con
estas funciones:

\begin{minted}{cpp}
  int set_field_opts(FIELD *field,          /* campo a modificar          */
                     int attr);             /* atributo a establecer      */

  int field_opts_on(FIELD *field,           /* campo a modificar          */
                    int attr);              /* atributo a encender        */

  int field_opts_off(FIELD *field,          /* campo a modificar          */
                     int attr);             /* atributo a apagar          */

  int field_opts(FIELD *field);             /* campo a consultar          */
\end{minted}

Puede utilizar la función set\_field\_opts() para establecer directamente
atributos de un campo o puede optar por encender/apagar algunos atributos con
field\_opts\_on() y field\_opts\_off() . En cualquier momento puede consultar los
atributos de un campo con field\_opts(). La siguiente es la lista de opciones
disponibles. De forma predeterminada, todas las opciones están encedidas.

\textbf{O\_VISIBLE}\\
Controla si el campo es visible en pantalla. Puede ser utilizado al procesar el
formulario para ocultar o mostar campos en función del valor de campos padre.\\

\textbf{O\_ACTIVE}\\
Controla si el campo está activo durante el procesamiento de formularios (es
decir, se puede visitar el formulario con las teclas de navegación). Se puede
utilizar para crear etiquetas o campos derivados con valores de buffer
modificados por la aplicación, no el usuario.\\

\textbf{O\_PUBLIC}\\
Controla si se muestran los datos durante lo entrada. Si se desactiva esta
opción en un campo, la libreria va a aceptar y editar datos en ese campo, pero
no se mostrarán y el cursor no se moverá. Puede apagar el bit O\_PUBLIC para
definir campos de contraseña.\\

\textbf{O\_EDIT}\\
Controla si los datos del campo pueden ser modificados. Cuando esta opción está
desactivada, fallaran todas las peticiones de edición excepto REQ\_PREV\_CHOICE y
REQ\_NEXT\_CHOICE. Estos campos de sólo lectura pueden ser útiles para mensajes
de ayuda.\\

\textbf{O\_WRAP}\\
Controla el ajuste de texto en campos multi-linea. Normalmente, cuando
cualquier carácter o (espacio separador) palabra llega al final de la línea
actual, se envuelve la palabra entera y salta a la línea siguiente (suponiendo
que la hay). Cuando esto opción está desactivada, la palabra se dividirá
atraves del salto de línea.\\

\textbf{O\_BLANK}\\
Controla la suprecion de caracteres. Cuando esta opción está activa, al
introducir un carácter en la primera posición del campo se borra todo el campo
(excepto el caracter ingresado).\\

\textbf{O\_AUTOSKIP}\\
Controla el salto automático al siguiente campo cuando el actual se llena.
Normalmente, cuando el usuario trata de ingresar mas datos en un campo hasta
llenarlo, la ubicación de edición salta al siguiente campo. Cuando esta opción
está desactivada, el cursor cuelga al final del campo. Esta opción se ignora en
los campos dinámicos que no han alcanzado su límite de tamaño.\\

\textbf{O\_NULLOK}\\
Controla si la validación se aplica a campos en blanco. Normalmente, no es asi;
el usuario puede dejar un campo en blanco sin invocar la comprobación de
validación al salir. Si esta opción está desactivada en un campo, la salida
invocará una comprobación de validación.\\

\textbf{O\_PASSOK}\\
Controla si la validación ocurre en cada salida, o sólo después de que el campo
es modificado. Normalmente es cierto esto último. Ajustar O\_PASSOK puede ser
útil si su funcion de validación puede cambiar durante el tratamiento de
formularios.\\

\textbf{O\_STATIC}\\
Controla si el campo se fija a su dimensiones iniciales. Si se desactiva, el
campo se vuelve dinámico y se extiende para adaptarse a los datos
introducidos.\\

Las opciones de un campo no se puede cambiar mientras el campo se encuentra
seleccionado. Sin embargo, se pueden cambiar las opciones en los demas campos.\\

Los valores de las opciones son mascaras de bits y se pueden componer con
puertas logicas OR de manera obvia. Usted ha visto el uso de apagar la opcion
O\_AUTOSKIP. El siguiente ejemplo aclara uso de algunas opciones más. Las otras
opciones se explicaran cuando sea apropiado.\\

\href{https://github.com/nasciiboy/NCURSES-Programming-HOWTO/blob/master/ncurses_programs/forms/form_options.c}{*Ejemplo 27. Opciones de Campo*}
\inputminted{cpp}{./cpp/027_opciones_campo.cpp}


En este ejemplo, aunque inútil, muestra el uso de opciones. Si se usan
adecuadamente, puede presentar la información de manera muy eficaz. El segundo
campo es no O\_PUBLIC, no muestra los caracteres que se están escribiendo.

\subsubsection{Estado del Campo}%
El estado de campo especifica si el campo ha sido editado o no. Inicialmente se
establece en FALSE y cuando el usuario escribe algo y los datos del buffer son
modificados se convierte en TRUE. Por lo tanto se puede consultar el estado de
un campo para saber si se ha modificado o no. Las siguientes funciones pueden
ayudar a esas operaciones.\\

\begin{minted}{cpp}
  int set_field_status(FIELD *field,      /* campo a modificar       */
                       int status);       /* estado a entablecer     */

  int field_status(FIELD *field);         /* obtener estado de campo */
\end{minted}

Es mejor comprobar el estado de un campo sólo después de salir de este, ya que
los datos del buffer podrian no haber sido actualizados aun, mientras la
validacion no se ha teminado. Para garantizar que es regresado el estado
correcto, llamade a field\_status() ya sea (1) cuando termina la rutina de
validacion del campo, (2) a partir de la inicializacion o terminacion de los
campos o formularios, o (3) enseguida de que una solicitud REQ\_VALIDATION ha
sido procesada por form\_driver().

\subsubsection{Puntero de Usuario de Campo}%
Cada estructura de campo contiene un puntero que puede ser utilizado por el
usuario para diversos fines. No es empleado por la libreria form y puede ser
utilizado por el usuario para cualquier propósito. Las siguientes funciones
establecen y obtienen el puntero del usuario.

\begin{minted}{cpp}
  int set_field_userptr(FIELD *field,
                        char *userptr);   /* el puntero de usuario que desea asociar */
                                          /* con el campo                            */

  char *field_userptr(FIELD *field);      /* obtener el puntero de usuario del campo */
\end{minted}

\subsubsection{Campos de Tamaño Variable}%
Si quiere un campo que cambie su anchura dinámicamente, esta es la
característica que desea poner a pleno uso. Esto permite que el usuario
introdusca más datos que el tamaño original del campo y deja crecer el campo.
De acuerdo con la orientación del campo lo hará desplazarse horizontal o
verticalmente para incorporar los nuevos datos.\\

Para obtener un campo con crecimiento dinámico, la debe apagar la opción
O\_STATIC. Esto se puede hacer con:

\begin{minted}{cpp}
  field_opts_off(field_pointer, O_STATIC);
\end{minted}

Pero por lo general no es aconsejable permitir a un campo para crecer
infinitamente. Puede establecer un límite máximo para el crecimiento con:

\begin{minted}{cpp}
  int set_max_field(FIELD *field,    /* Campo sabre el cual operar                 */
                    int max_growth); /* Crecimiento maximo permitido para el campo */
\end{minted}

La informacion de un campo con crecimiento dinamico se puede recuperar utilizando:

\begin{minted}{cpp}
  int dynamic_field_info( FIELD *field,     /* campo sobre el cual operar   */
                          int   *prows,     /* numero de filas              */
                          int   *pcols,     /* number de columnas           */
                          int   *pmax)      /* maximo crecimiento permitido */
\end{minted}

Aunque field\_info funciona como de costumbre, es aconsejable utilizar esta
función para obtener los atributos correctos de un campo con crecimiento
dinamico. Recordemos la rutina new\_field de libreria; un campo creado con la
altura establecida en uno define un campo de una sola línea. Un campo creado
con una altura mayor a uno define un campo de múltiples líneas.\\

Un campo de una línea con O\_STATIC apagado (campo con crecimiento dinamico)
contendrá una sola fila fija, pero el número de columnas puede aumentar si el
usuario introduce más datos que los fijados originalmente. El número de
columnas mostradas permanecerá fijo y los datos adicionales podran desplazarse
horizontalmente.\\

Un campo multilínea con O\_STATIC apagado (campo con crecimiento dinamico)
contendrá un número fijo de columnas, pero el número de filas puede aumentar si
el usuario introduce más datos que los fijados originalmnte. El número de filas
que se muestra permanecerá fijo y los datos adicionales podran desplazarse
verticalmente.\\

Los dos párrafos anteriores describen más o menos el comportamiento de un campo
dinamico. La forma en que se comportan otras partes de libreria se describen a
continuacion:\\

\begin{itemize}
  \item La opción de campo O\_AUTOSKIP se ignorará si O\_STATIC está apagado y no
    se ha especificado un crecimiento máxima para el campo. Actualmente,
    O\_AUTOSKIP genera automaticamente una solicitud REQ\_NEXT\_FIELD cuando el
    usuario escribe sobre la posición del ultimo carácter de un campo. En un
    campo sin crecimiento máximo especificado, no hay un último carácter. Si se
    especifica un crecimiento máximo, la opcion O\_AUTOSKIP funcionará de forma
    normal si el campo ha crecido a su máximo tamaño.
  \item Se ignora la justificación de campo si se ha apagado la opción
    O\_STATIC. Actualmente se puede utilizar, set\_field\_just con JUSTIFY\_LEFT,
    JUSTIFY\_RIGHT, JUSTIFY\_CENTER para los contenidos de un campo de una línea.
    Un campo creciente de una linea, por definición, crece y se desplaza
    horizontalmente y puede contener más datos de los que se pueden justificar.
    El retorno de field\_just será NO\_JUSTIFICATION.
  \item Sobrecargar form\_driver con la solicitud REQ\_NEW\_LINE operará de la
    misma manera independientemente de la opción O\_NL\_OVERLOAD si la opción de
    campo O\_STATIC está apagada y no hay un crecimiento máximo especificado
    para el campo. Actualmente, si la opción O\_NL\_OVERLOAD está encendida,
    REQ\_NEW\_LINE genera implícitamente una solicitud REQ\_NEXT\_FIELD si se llama
    desde la última línea de un campo. Si una campo puede crecer sin límite, no
    hay última línea, por lo que REQ\_NEW\_LINE nunca generará implícitamente una
    solicitud REQ\_NEXT\_FIELD. Si se especifica un límite máximo de crecimiento
    y se activa la opcion O\_NL\_OVERLOAD, REQ\_NEW\_LINE sólo se generar
    REQ\_NEXT\_FIELD si el campo ha crecido hasta su tamaño máximo y el usuario
    está en la última línea.
  \item La funcion dup\_field trabajara como de costumbre; duplicara el campo,
    incluyendo el tamaño del búfer actual y el contenido del campo. Cualquier
    crecimiento máximo especificado también se duplicará.
  \item La llamada link\_field funcionará como de costumbre; duplicara todos los
    atributos de campo y compartira buffers con el campo enlazado. Si la opcion
    de campo O\_STATIC se cambia posteriormente por buffers de campo compatidos,
    la reaccion del sistema a un intento de introducir más datos en el campo
    actualmente fijo dependerá de la configuración de la opción en el campo
    actual.
  \item La llamada field\_info funcionará como de costumbre; la variable nrow
    contendrá el valor de la llamada original a new\_field. El usuario debe
    utilizar dynamic\_field\_info, descrito anteriormente, para consultar el
    tamaño actual del buffer.
\end{itemize}

Algunos de los puntos anteriores tienen sentido sólo después de explicar
form\_driver. Cosa que buscaremos hacer en próximas secciones.

\subsection{Ventanas de Formulario}% 
El concepto de ventana de formulario es más o menos similar al de ventana de
menu. Cada formulario está asociada con una ventana principal y una sub
ventana. La ventana principal del formulario muestra cualquier título o borde
asociado o lo que sea que desee. La sub ventana contiene todos los campos y los
muestra en función de su posición. Esto le da la flexibilidad de manipular la
presentacion de los formularios muy fácilmente.\\

Dado que esto es muy similar a las ventanas de menu, proporciono un ejemplo con
una mayor explicación. Las funciones son similares y funcionan de la misma
manera.\\

\href{https://github.com/nasciiboy/NCURSES-Programming-HOWTO/blob/master/ncurses_programs/forms/form_win.c}{*Ejemplo 28. Ventanas de Formulario}
\inputminted{cpp}{./cpp/028_ventanas_formulario.cpp}

\subsection{Validación de Campo}%
Por defecto, un campo aceptará cualquier entrada de datos por parte del
usuario. Es posible colocar la validación en el campo. Por tanto, fallará
cualquier intento por parte del usuario de abandonar el campo, mientras este
contiene datos que no concuerdan con el tipo de validación. Algunos tipos de
validación también tienen validacion por carácter para verificar cada vez que
se introduce un carácter en el campo.\\

La validación a un campo se pude agregar con la siguiente función.

\begin{minted}{cpp}
  int set_field_type(FIELD *field,          /* campo a modificar      */
                     FIELDTYPE *ftype,      /* tipo asociado          */
                     ...);                  /* argumentos adicionales */
\end{minted}

Una vez establecido, el tipo de validación de un campo se puede consultar con:

\begin{minted}{cpp}
  FIELDTYPE *field_type(FIELD *field);      /* campo a consultar       */
\end{minted}

El controlador de formulario valida los datos en un campo solo cuando los datos
son ingresados por el usuario final. La Validación no se produce cuando:

\begin{itemize}
  \item El programa de aplicación cambia el valor del campo llamando a
    set\_field\_buffer.
  \item Vincular valores de campo y cambiarlos indicertamente – cambiando el
    campo al que está vinculado
\end{itemize}

Los siguientes son los tipos de validación predefinidos. También puede
especificar una validación personalizada, aunque es un poco complicado y
engorroso.

\subsubsection*{TYPE\_ALPHA}%
Este tipo de campo acepta datos alfabéticos; sin espacios, digitos o caracteres
especiales (esto se comprueba en cada ingreso de caracter). Este se crea con:

\begin{minted}{cpp}
  int set_field_type(FIELD *field,          /* campo a modificar     */
                     TYPE_ALPHA,            /* tipo asociado         */
                     int width);            /* ancho minimo de campo */
\end{minted}

El argumento width establece una anchura minima de datos. El usuario tiene que
ingresar un número de al menos el ancho de caracteres antes poder salir del
campo. Normalmente, querrá establecer esto a la anchura del campo; si es mayor
que el ancho del campo, la comprobación de validación siempre fallara. Una
anchura menor a cero crea un campo con terminación opcional.

\subsubsection*{TYPE\_ALNUM}%
Este tipo de campo acepta datos alfanumericos; sin espacios en blanco, ni
caracteres especiales (esto se comprueba en cada ingreso de caracter). Este se
crea con:

\begin{minted}{cpp}
  int set_field_type(FIELD *field,          /* campo a modificar     */
                     TYPE_ALNUM,            /* tipo asociado         */
                     int width);            /* ancho minimo de campo */
\end{minted}

El argumento width establece una anchura mínima de datos. Al igual que
TYPE\_ALPHA, normalmente, querrá establecer esto a la anchura del campo; si es
mayor que el ancho del campo, la comprobación de validación siempre fallara.
Una anchura menor a cero crea un campo con terminación opcional.

\subsubsection*{TYPE\_ENUM}%
Este tipo le permite restringir los valores de un campo a estar entre un
conjunto de valores de cadena especifico (por ejemplo, el codigo postal de dos
letras de los estados de Estados Unidos). Se crea con:

\begin{minted}{cpp}
  int set_field_type(FIELD *field,          /* campo a modificar         */
                     TYPE_ENUM,             /* tipo asociado             */
                     char **valuelist;      /* lista de valores posibles */
                     int checkcase;         /* sensible a mayusculas?    */
                     int checkunique);      /* coincidencia unica?       */
\end{minted}

El parámetro valuelist debe apuntar a una lista de cadenas válidas terminada en
NULL. Si el argumento checkcase, es verdadero, hace la comparación sensible a
mayúsculas y minúsculas.\\

Cuando el usuario sale de un campo TYPE\_ENUM, el procedimiento de validación
intenta completar los datos en el buffer a una entrada válida. Si ha ingresado
una cadena completa de elección, por supuesto es válido. Pero también es
posible introducir un prefijo de una cadena válido y que se complete para
usted.\\

Por defecto, si se introduce un prefijo y coincide con más de un valor en la
lista de cadenas, el prefijo se completará al primer valor coincidente. Pero el
argumento checkunique, de ser cierto, requiere que el prefijo coincida con un
unico valor para ser válido.\\

Las peticiones REQ\_NEXT\_CHOICE y REQ\_PREV\_CHOICE pueden ser particularmente
útiles con estos campos.

\subsubsection*{TYPE\_INTEGER}%
Este tipo de campo acepta un entero. Se configura de la siguiente manera:

\begin{minted}{cpp}
  int set_field_type(FIELD *field,          /* campo a modificar           */
                     TYPE_INTEGER,          /* tipo asociado               */
                     int padding,           /* precision de relleno a cero */
                     int vmin, int vmax);   /* rango valido                */
\end{minted}

Los caracteres válidos consisten en un menos inicial opcional y dígitos. La
prueba de rango se realiza a la salida. Si el rango máximo es menor o igual al
mínimo, se ignora.\\

Si el valor pasa la prueva de rango, se rellena con ceros a la izquierda hasta
cumplir con el argumento padding.\\

Un buffer de valor TYPE\_INTEGER es interpretado convenientemente con la funcion
de la libreria C atoi(3).

\subsubsection*{TYPE\_NUMERIC}%
Este tipo de campo acepta un número decimal. Se configura de la siguiente
manera:

\begin{minted}{cpp}
  int set_field_type(FIELD *field,          /* campo a modificar           */
                     TYPE_NUMERIC,          /* tipo asociado               */
                     int padding,           /* procicion de relleno a cero */
                     int vmin, int vmax);   /* rango valido                */
\end{minted}

Los caracteres válidos consisten en un menos inicial opcional y dígitos,
posiblemente incluyendo un punto decimal. La prueba de rango se realiza a la
salida. Si el rango máximo es menor o igual al mínimo, se ignora.\\

Si el valor pasa la prueva de rango, se rellena con ceros a la izquierda hasta
cumplir con el argumento padding.\\

Un buffer de valor TYPE\_NUMERIC es interpretado convenientemente con la funcion
de la libreria C atof(3).

\subsubsection*{TYPE\_REGEXP}%
Este tipo de campo acepta datos que coincidan con una expresión regular. Se
configura de la siguiente manera:

\begin{minted}{cpp}
  int set_field_type(FIELD *field,          /* campo a modificar   */
                     TYPE_REGEXP,           /* tipo asociado       */
                     char *regexp);         /* exprecion a empatar */
\end{minted}

La sintaxis de las expresiones regulares es la de regcomp(3). La pueba de
coincidencia de la expresión regular se ejecuta a la salida.

\subsection{Controlador del Formulario: El caballo de trabajo del sistema de formularios}%
Al igual que en el sistema de menus, form\_driver() juega un papel muy
importante en el sistema de formularios. Todos los tipos de peticiones al
sistema de formularios deben ser canalizado a través form\_driver().

\begin{minted}{cpp}
  int form_driver(FORM *form,     /* formulario en el que operar */
                  int request)    /* codigo de solicitud         */
\end{minted}

Como se ha visto en ejemplos anteriores, se tiene que estar en un bucle en
busca de entrada del usuario y luego decidir si se trata de una solicitud de
campo o una de formulario. Las solicitudes de formulario se pasan a
form\_driver() para hacer el trabajo.

Se pueden dividir las peticiones aproximadamente en las siguientes categorías.
Las diferentes solicitudes y su uso se explica a continuación:

\subsubsection{Solicitudes de Navegacion por Pagina}%
Estas solicitudes causan que el nivel de página se mueve a través de la
formulario, lo que provoca la visualización de una pantalla nueva de
formulario. Una formulario puede estar hecho de múltiples páginas. Si tiene una
formulario con gran cantidad de campos y secciones lógicas, puede dividir el
formulario en páginas. La función set\_new\_page() para establecer una página
nueva en un campo especifico.

\begin{minted}{cpp}
  int set_new_page(FIELD *field,        /* Campo en el cual poner o quitar el salto de pagina */
                   bool new_page_flag); /* debe ser TRUE para colocar el salto                */
\end{minted}

Las siguientes peticiones le permiten moverse a diferentes páginas:

\begin{itemize}
  \item REQ\_NEXT\_PAGE mover a la página siguiente.
  \item REQ\_PREV\_PAGE mover a la página anterior.
  \item REQ\_FIRST\_PAGE mover a la primer página.
  \item REQ\_LAST\_PAGE mover a la última página.
\end{itemize}

Estas solicitudes manejan la lista de forma cíclica; es decir, desde la ultima
pagina REQ\_NEXT\_PAGE va a la primera, y desde la primer pagina REQ\_PREV\_PAGE va
a la última.

\subsubsection{Solicitudes de Navegación Entre Campos}%
Estas solicitudes manejar la navegación entre los campos de una misma página.

\begin{itemize}
  \item REQ\_NEXT\_FIELD mover al campo siguiente.
  \item REQ\_PREV\_FIELD mover al campo anterior.
  \item REQ\_FIRST\_FIELD mover al primer campo.
  \item REQ\_LAST\_FIELD Mover al último campo.
  \item REQ\_SNEXT\_FIELD Mover al siguiente campo ordenado.
  \item REQ\_SPREV\_FIELD Mover al anteriar campo ordenado.
  \item REQ\_SFIRST\_FIELD mover al primer campo ordenado.
  \item REQ\_SLAST\_FIELD Mover al último campo ordenado.
  \item REQ\_LEFT\_FIELD Mover a la izquierda del campo.
  \item REQ\_RIGHT\_FIELD Mover a la derecha del campo.
  \item REQ\_UP\_FIELD mover hacia arriba un campo.
  \item REQ\_DOWN\_FIELD mover hacia abajo un campo.
\end{itemize}

Estas peticiones manejan la lista de campos en una página de forma cíclica; ese
es decir, desde el ultimo campo REQ\_NEXT\_FIELD va a el primero, y desde el
primer campo REQ\_PREV\_FIELD va a el ultimo. El orden de los campos para estas
peticiones (incluidas REQ\_FIRST\_FIELD y REQ\_LAST\_FIELD) es simplemente el orden
de los punteros en el arreglo de formularios (establecido por new\_form() o
set\_form\_field())\\

También es posible recorrer los campos como si hubieran sido ordenados
deacuerdo a su posicion en pantalla, por lo que la secuencia va de izquierda a
derecha y de arriba a abajo. Para ello, utilice el segundo grupo de cuatro
movimientos por orden.\\

Por último, es posible moverse entre campos utilizando las direciones arriba,
abajo, derecha, e izquierda. Para lograr esto, utilice el tercer grupo de
cuatro movimientos. Sin embargo, observe que la posición de un formulario a
efectos de estas peticiones es la esquina superior izquierda.\\

Por ejemplo, suponga que tiene un campo B multilinea, y dos campos de una sola
línea A y C en la misma línea con B, A a la izquierdo de B y C a la derecha de
B. Desde A REQ\_MOVE\_RIGHT va a B sólo si A, B, y C comparten la misma primera
línea; de lo contrario, saltar a C.

\subsubsection{Solicitudes de Navegacion dentro del Campo}%
Estas solicitudes manejan el movimiento de el cursor de edición dentro del
campo artalmente seleccionado.

\begin{itemize}
  \item REQ\_NEXT\_CHAR Mover al carácter siguiente.
  \item REQ\_PREV\_CHAR Mover a carácter anterior.
  \item REQ\_NEXT\_LINE Mover a la línea siguiente.
  \item REQ\_PREV\_LINE Mover a la línea anterior.
  \item REQ\_NEXT\_WORD Mover a la palabra siguiente.
  \item REQ\_PREV\_WORD Mover a la palabra anterior.
  \item REQ\_BEG\_FIELD Mover a inicio del campo.
  \item REQ\_END\_FIELD Mover a final del campo.
  \item REQ\_BEG\_LINE Mover al incio de línea.
  \item REQ\_END\_LINE Mover a final de línea.
  \item REQ\_LEFT\_CHAR Mover a la izquierda en el campo.
  \item REQ\_RIGHT\_CHAR Mover a la derecha en el campo.
  \item REQ\_UP\_CHAR Mover hacia arriba en el campo.
  \item REQ\_DOWN\_CHAR Mover hacia abajo en el campo.
\end{itemize}

Cada palabra se separa de los caracteres anteriores y siguientes por espacio en
blanco. Los comandos para mover al inicio y fin de línea o camo buscan el
primer o el ultimo caracter que no es de relleno en sus rangos.

\subsubsection{Solicitudes de Desplazamiento}%
Los campos dinámicos y los creados explícitamente con filas fuera de pantalla
son desplazables. Los campos de una linea se desplazan horizontalmente; los
campos multilínea verticalmente. La mayoría de movimientos se acciona por la
edición y el movimiento dentro del campo (la libreria desplaza el campo para
mantener el cursor visible). Es posible solicitar explícitamente el
desplazamiento con las siguientes solicitudes:

\begin{itemize}
  \item REQ\_SCR\_FLINE Desplazar verticalmente hacia adelante una línea.
  \item REQ\_SCR\_BLINE Desplazar verticalmente hacia atrás una línea.
  \item REQ\_SCR\_FPAGE Desplazar verticalmente hacia adelante una página.
  \item REQ\_SCR\_BPAGE Desplazar verticalmente hacia atrás una página.
  \item REQ\_SCR\_FHPAGE Desplazar verticalmente hacia adelante media página.
  \item REQ\_SCR\_BHPAGE Desplazar verticalmente hacia atrás media página.
  \item REQ\_SCR\_FCHAR Desplazar horizontal hacia adelante un caracter.
  \item REQ\_SCR\_BCHAR Desplazar horizontal hacia atrás un caracter.
  \item REQ\_SCR\_HFLINE Desplazar horizontal el ancho de campo hacia adelante.
  \item REQ\_SCR\_HBLINE Desplazar horizontal el ancho de campo hacia atrás.
  \item REQ\_SCR\_HFHALF Desplazar horizontal medio ancho de campo hacia adelante.
  \item REQ\_SCR\_HBHALF Desplazar horizontal medio ancho de campo hacia atras.
\end{itemize}

A efectos de desplazamiento, la página de un campo es la altura de su parte visible.

\subsubsection{Solicitudes de Edición}%
Cuando se pasa un caracter ASCII al controlador de formulario, se trata como
una solicitud para agregar el carácter a buffer de datos del campo. Si es un
replazo o una inserción depende del modo de edición del campo (por defecto es
inserción).\\

Las siguientes solicitudes apoyan la edicion del campo y cambian de modo de
edición:

\begin{itemize}
  \item REQ\_INS\_MODE Establece modo en inserción.
  \item REQ\_OVL\_MODE Establece modo de superposición.
  \item REQ\_NEW\_LINE Solicita nueva línea (ver más abajo para la explicación).
  \item REQ\_INS\_CHAR Insertar espacio en la posicion de carácter.
  \item REQ\_INS\_LINE Insertar línea en blanco en la posición del carácter.
  \item REQ\_DEL\_CHAR Eliminar el carácter en el cursor.
  \item REQ\_DEL\_PREV Eliminar palabra anterior al cursor.
  \item REQ\_DEL\_LINE Eliminar linea en el cursor.
  \item REQ\_DEL\_WORD Eliminar la palabra en el cursor.
  \item REQ\_CLR\_EOL Borrar hasta el final de la línea.
  \item REQ\_CLR\_EOF Borrar hasta el final del campo.
  \item REQ\_CLR\_FIELD Borrar campo entero.
\end{itemize}

El comportamiento de las peticiones REQ\_NEW\_LINE y REQ\_DEL\_PREV es complicado y
en parte controlado por un par de opciones del formulario. Los casos especiales
se activan cuando el cursor se encuentra al inicio o en la última línea del
campo.\\

En primer lugar, consideramos REQ\_NEW\_LINE:\\

El comportamiento normal de REQ\_NEW\_LINE en modo de inserción es romper la
línea actual en la posición del cursor, colocando el resto de linea despues del
cursor como una nueva línea y moviendo el cursor al inicio de la nueva línea
(puede pensar en esto como la inserción de una nueva línea en el buffer de
campo).\\

El comportamiento normal de REQ\_NEW\_LINE en el modo de superposición es limpiar
la línea actual desde la posición del cursor al fin de línea. El cursor se
mueve entonces hasta el inicio de la línea siguiente.\\

Sin embargo, REQ\_NEW\_LINE al incio de un campo, o sobre la última línea de un
campo, en cambio realiza REQ\_NEXT\_FIELD. Si la opcion O\_NL\_OVERLOAD esta
desactivada, se desactiva esta acción especial.\\

Ahora, consideremos REQ\_DEL\_PREV:\\

El comportamiento normal de REQ\_DEL\_PREV es eliminar el caracter anterior. Si
el modo de inserción está activado, y el cursor se encuentra al inicio de una
línea, y el texto en esa línea cabe en la anterior, se anexa el contenido de la
línea actual a la anterior y se borra la línea actual (puede pensar en esto
como la eliminación de un salto de línea del buffer de campo).\\

Sin embargo, REQ\_DEL\_PREV al inicio de un campo se trata como REQ\_PREV\_FIELD.\\

Si la opcion O\_BS\_OVERLOAD está desactivada, se desactiva esta acción especial
y el controlador devuelve sólo E\_REQUEST\_DENIED.

\subsubsection{Solicitudes de Peticion}%
Si puede solicitar el contenido de un campo, y posse funciones asociadas para
obtener los valores siguiente y anterior de tipo de un valor dado, estas son
las solicitudes para obtener este valor dentro del buffer del campo:

\begin{itemize}
  \item REQ\_NEXT\_CHOICE Coloca el volor succesor del actual valor en el buffer.
  \item REQ\_PREV\_CHOICE Coloca el valor predecesor del actual valor en el buffer.
\end{itemize}

De los tipos de campo incorporados, solamente TYPE\_ENUM ha incorporado
funciones de sucesor y predecesor. Cuando define un tipo de campo por usted
mismo (ver Validacion de Tipos Personalizados), puede asociar su propia
funciones de peticion.

\subsubsection{Comandos de aplicacion}%
Las peticiones de formulario se representan como números enteros por encima del
valor KEY\_MAX y debajo o igual a la constante MAX\_COMMAND. Un valor dentro de
este rango es ignorado por form\_driver(). Así que se puede utilizar para
cualquier propósito de la aplicacion. Se puede tratar como una acción
específica de la aplicación y tomar lao accion correspondiente.

\section{Librerias de Widgets y Herramientas}%
Ahora que usted ha visto las capacidades de ncurses y sus librerias hermanas,
se esta enrollando las mangas preparandose para un proyecto con una fuerte
manipulacion de pantalla. Pero espere.. Puede ser muy difícil escribir y
mantener widgets complejos con puro ncurses o incluso con las librerias
adicionales. Puede utilizar algunas librerias de widgets y herramientas listas
para usar en lugar de escribir su propios widgets. Puede utilizar algunos de
ellos, obtener ideas a partir del código, o incluso extenderlas.

\subsection{CDK (Kit de desarrollo de curses)}%
En palabras del autor\\

El kit proporciona algunos widgets útiles, que pueden utilizarse directamente
en sus programas. Está muy bien escrito y la documentación es muy buena. Los
ejemplos en el directorio examples pueden ser un buen punto de partida para los
principiantes. CDK se puede descargar desde http://invisible-island.net/cdk/.
Siga las instrucciones del archivo README para instalarlo.

\subsubsection{Lista de Widgets}%
La siguiente es la lista de widgets provista de cdk y su descripción.

\begin{itemize}
  \item Alphalist: Permite seleccionar al usuario de una lista de palabras, con
    la capacidad de reducir la lista de búsqueda escribiendo algunos caracteres
    de la palabra deseada.
  \item Buttonbox: Crea widgets de multiples botones.
  \item Calendar: Crea un pequeño y simple widget calendario.
  \item Dialog: Informa un mesaje al usuario, y el usuario puede escojer una obcion de los botones proporcionados.
  \item Entry: Permite al usuario ingresar varios tipos de datos.
  \item File Selector: Un selector de ficheros construido con widgets CDK. Esta
    ejemplo muestra cómo crear widgets más complicados utilizando la librería
    CDK.
  \item Graph: Dibuja un grafico.
  \item Histogram: Dibuja un histograma.
  \item Item List: Crea un campo emergente que permite al usuario seleccionar
    una de las varias opciones en un pequeño campo. Muy útil para cosas como
    día de la semana o nombre de mes.
  \item Label: Muestra mensajes sobre una caja emergente, o sobre etiqueta que se considera parte de la pantalla.
  \item Marquee: Muestra un mensaje sobre una marquesina con desplazamiento.
  \item Matrix: Crea una compleja matrix con muchas opciones.
  \item Menu: Crea una interfaz de menu desplegable.
  \item Multiple Line Entry: Un campo multilinea de entrada. Muy útil para los
    campos largos. (como un campo de descripción)
  \item Radio List: Crea una lista de radio botones.
  \item Scale: Crea una escala numérica. Se utiliza para permitir a un usuario
    elegir un valor numérico y restringirlos a una serie de valores.
  \item Scrolling List: Crea un menu/lista desplazable.
  \item Scrolling Window: Crea un visor de registros desplazable. Puede agregar
    información en la ventana mientras se ejecuta. Un buen widget para
    visualizar el progreso de algo. (similar a una ventana de la consola)
  \item Selection List: Crea una lista de opciones de seleccion multiple.
  \item Slider: Similar al widget escale, este widget proporciona una barra
    deslizante visual para representar el valor numérico.
  \item Template: Crea un campo de entrada con caracteres sensibles a una
    posicion. Se utiliza para pre-formatear campos como fechas y numeros de
    telefono.
  \item Viewer: Este es un visor de ficheros/informacion. Muy util cuando se
    necesita mostrar un monton de informacion.
\end{itemize}

Algunos de los widgets son modificados por Thomas Dickey en las últimas versiones.

\subsubsection{Algunas Características Atractivas}%
Ademas de hacer nuestra vida más fácil con widgets fácilmente utilizables, cdk
resuelve un problema frustrante con la impresión de cadenas de colores, cadenas
justificadas elegantemente. Las etiquetas con formato especial se puede
incrustar en las cadenas pasadas a las funciones CDK. por Ejemplo

Si la cadena
\begin{minted}{cpp}
  "</B/1>Esta line debe tener un primer plano amarillo y un fondo azul.<!1>"
\end{minted}

se pasa como parámetro a newCDKLabel(), se imprime la línea con primer plano
amarillo y fondo azul. Hay otras etiquetas disponibles para justificar la
cadeno, dibujo de caracteres especiales etc.. Por favor, consulte la página man
cdk\_display(3X) para mas detalles. La página del manual explica el uso con
buenos ejemplos.

\subsubsection{Conclusión}%
Considerando todo, CDK es un paquete de widgets bien escrito, que si se utiliza
correctamente puede formar un fuerte marco de trabajo para el desarrollo de
complejos GUI.

\subsection{dialog}%
Hace mucho mucho tiempo, en Septiembre de 1994, cuando pocas personas sabían
GNU/linux, Jeff Tranter escribió un articulo de diálog en Linux Journal. El
articulo comienza con estas palabras..\\

Como él mismo explica, dialog es una verdadera joya para crear cajas de dialogo
de aspecto profesional con facilidad. Crea una variedad de cuadros de diálogo,
menus, listas de verificación, etc.. Por lo general se instala de forma
predeterminada. Si no, puede descargarlo desde el sitio de Thomas Dickey’s.\\

El artículo mencionado da una muy buena descripción de sus usos y capacidades.
La página del man tiene más detalles. Se puede utilizar en variedad de
situaciones. Un buen ejemplo es la construcción del kernel linux en modo texto.
El kernel linux utiliza una versión de dialog modificada a medida para sus
necesidades.\\

dialog fue diseñado inicialmente para ser utilizado con shell scripts. Si desea
utilizar su funcionalidad en un programa c, entonces puede utilizar libdialog.
La documentación en este aspecto es escasa. La referencia definitiva es el
archivo de cabecera dialog.h que viene con el libreria. Puede que tenga que
cortar aquí y allá para conseguir la salida necesaria. El codigo fuente es
fácilmente personalizable. Lo he usado en un número de ocasiones mediante la
modificación del código.

\subsection{Módulos curses de Perl CURSES::FORM y CURSES::WIDGETS}%
Los módulos curses de perl, curses::Form y curses::Widgets dan acceso a curses
desde perl. Si usted tiene la instalacion basica de curses y perl, puede
obtener estos módulos de CPAN All Modules page. Obtener los tres módulos
comprimidos en la categoría curses. Una vez instalados puede utilizar estos
módulos en scripts de Perl como cualquier otro módulo. Para obtener más
información sobre los módulos vea la pagina man perlmod. Los módulos anteriores
vienenn con una buena documentación y tienen algunos scripts de demostración
para probar su funcionalidad. Aunque los widgets proporcionadas son muy
rudimentarios, estos módulos proporcionan un buen acceso a la libreria curses
desde perl.\\

Algunos de mis ejemplos de código fueron adaptados a perl por Anuradha
Ratnaweera y están disponibles en el directorio perl.\\

Para obtener más información, consulte las páginas de manual curses(3),
curses::Form(3) y curses::Widgets(3). Estas paginas se instalan sólo cuando los
módulos anteriores se adquieren e instalan.

\section{Solo Por Diversion !!!}%
Esta sección contiene algunos programas escritos por mí sólo por diversión.
Esto no significa que sean una mejor práctica de programación o la mejor manera
de utilizar ncurses. Se proporcionan a fin de permitir que los principiantes
obtegan ideas y añaden más programas a esta sección. Si usted ha escrito un par
de buenos programas, programas sencillos en curses y los quiere incluir aquí,
contacteme.

\subsection{El juego de la vida}%
El Juego de la vida es una maravilla de la matemáticas. En palbras de Paul
Callahan.\\

{\texttt El juego de la vida (o simplemente la vida) no es un juego en el sentido
convencional. Ya que tiene jugadores, y no se gana o pierde. Una vez que las
"piezas" se colocan en la posición inicial, las reglas determinan todo lo que
sucede después.  Sin embargo, la vida está llena de sorpresas! En la mayoría de
los casos, es imposible mirar un posición de inicial (o patrón) y ver lo que
sucederá en el futuro. La única manera de averiguarlo es seguir las reglas del
juego.}\\

Este programa comienza con un simple patrón o recambio y muestra el maravilloso
trabajo de la vida. Hay mucho margen de mejora en el programa. Puede dejar que
el usuario introduzca el patrón de su elección o incluso obtener información de
un archivo. También puede cambiar las reglas y jugar con un gran cantidad de
variaciones. Realice una busqueda en google para obtener interesante
información sobre el juego de vida.\\

\href{https://github.com/nasciiboy/NCURSES-Programming-HOWTO/blob/master/ncurses_programs/JustForFun/life.c}{Game of life}
\inputminted{cpp}{./cpp/life.cpp}

\subsection{Cuadrado Mágico}%
Otra maravilla de las matemáticas, es muy sencillo de entender pero muy difícil
de hacer. En un cuadrado mágico la suma de los números en cada fila, y cada
columna es igual. Incluso la suma en digonal puede ser igual. Hay muchas
variaciones con propiedades particulares.\\

Este programa crea un sencillo cuadrado mágico de orden impar.

\href{https://github.com/nasciiboy/NCURSES-Programming-HOWTO/blob/master/ncurses_programs/JustForFun/magic.c}{Cuadrado magico}
\inputminted{cpp}{./cpp/magic.cpp}

\subsection{Torres de Hanoi}%
Solucionador de las famosas torres de hanoi. El objetivo del juego es mover los
discos de la primer a la último estaca, utilizando la estaca central como
temporal. No se permite tomar un disco y colocarlo sobre otro mas pequeño.\\

\href{https://github.com/nasciiboy/NCURSES-Programming-HOWTO/blob/master/ncurses_programs/JustForFun/hanoi.c}{Torres de Hanoi}
\inputminted{cpp}{./cpp/hanoi.cpp}

\subsection{Rompecabezas de las Reinas}%
El objetivo del famoso rompecabezas N-reina es poner N reinas en un tablero de
ajedrez N X N atacarse unas a otras.\\

Este programa resuelve esto con una técnica de rastreo hacia atras simple.\\

\href{https://github.com/nasciiboy/NCURSES-Programming-HOWTO/blob/master/ncurses_programs/JustForFun/queens.c}{Rompecabezas de las Reinas}
\inputminted{cpp}{./cpp/queens.cpp}

\subsection{Barajar}%
Un juego divertido, si tienes tiempo para matar.

\href{https://github.com/nasciiboy/NCURSES-Programming-HOWTO/blob/master/ncurses_programs/JustForFun/shuffle.c}{Barajar}
\inputminted{cpp}{./cpp/shuffle.cpp}

\subsection{Tutor de Mecanografia}%
Un tutor de mecanografía simple, Lo he creado más por necesidad que por su
facilidad de uso. Si sabe cómo colocar los dedos correctamente en el teclado,
pero le falta práctica, puede serle útil.
\href{https://github.com/nasciiboy/NCURSES-Programming-HOWTO/blob/master/ncurses_programs/JustForFun/tt.c}{Tutor de Mecanografía}
\inputminted{cpp}{./cpp/tt.cpp}

\section{Referencias}%
\begin{itemize}
  \item \href{https://github.com/nasciiboy/NCURSES-Programming-HOWTO/blob/master/man/index.org}{Páginas man} de NCURSES.
  \item NCURSES FAQ en \href{http://invisible-island.net/ncurses/ncurses.faq.html}{http://invisible-island.net/ncurses/ncurses.faq.html}
  \item Escribir programas con NCURSES por Eric Raymond y Zeid M. Ben-Halim en\\
    \href{http://invisible-island.net/ncurses/ncurses-intro.html}{http://invisible-island.net/ncurses/ncurses-intro.html}
    - algo obsoleto. Me inspiré en este documento y la estructura de este HOWTO
    sigue la del documento original
\end{itemize}





\vspace{2cm}
\LARGE\textit{RuneCode}


\end{document}

