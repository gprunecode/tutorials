%%
%% (
%%  )\ )                             (
%%  (()/(   (            (             )\  )   (
%%   /(_))  ))\   (       ))\  (   (   (()/(   ))\
%%   (_))  /((_)  )\  )  /((_) )\  )\   ((_))/((_)
%%   | _ \(_))(  _(_/( (_) )  ((_)((_)  _| |(_))
%%   |   /| || || ' \))/ -_)/ _|/ _ \/ _` |/ -_)
%%   |_|_\ \_,_||_||_| \___|\__|\___/\__,_|\___|
%%

\documentclass{article}
\usepackage[utf8]{inputenc}
\usepackage{amsmath}
%\usepackage{slashbox}
\usepackage{amsfonts}
\usepackage{amssymb}
\usepackage{graphicx} % Paquete para incluir imágenes en el documento LaTeX
\usepackage{hyperref}
\hypersetup{
  colorlinks=true,
  linkcolor=blue,
  filecolor=magenta,
  urlcolor=cyan,
}
\urlstyle{same}
\usepackage{varwidth}

\newcommand\tab[1][1cm]{\hspace*{#1}}

\usepackage{multirow}

\usepackage[a4paper,rmargin=1.5cm,lmargin=1.5cm,top=1.5cm,bottom=1.5cm]{geometry}

\usepackage{pdfpages}

\usepackage{xcolor}
\usepackage{minted}
\setminted[css]{frame=lines, framesep=2mm, baselinestretch=1.2, rulecolor=\color{black!80},
                bgcolor=DarkGray,fontsize=\normalsize}
\usemintedstyle[css]{monokai}
\setminted[python]{frame=lines, framesep=2mm, baselinestretch=1.2, rulecolor=\color{black!80}, bgcolor=DarkGray}
\usemintedstyle[python]{monokai}
\setminted[java]{frame=lines, framesep=2mm, baselinestretch=1.2, rulecolor=\color{black!80}, bgcolor=DarkGray}
\usemintedstyle[java]{monokai}
\setminted[javascript]{frame=lines, framesep=2mm, baselinestretch=1.2, rulecolor=\color{black!80}, bgcolor=DarkGray}
\usemintedstyle[javascript]{monokai}
\setminted[php]{frame=lines, framesep=2mm, baselinestretch=1.2, rulecolor=\color{black!30}, bgcolor=LightGray}
\setminted[html]{frame=lines, framesep=2mm, baselinestretch=1.2, rulecolor=\color{black!30}, bgcolor=LightGray}
\setminted[bash]{baselinestretch=1.2,rulecolor=\color{black!30},bgcolor=LightGray}
\definecolor{LightGray}{gray}{0.98}
\definecolor{DarkGray}{gray}{0.1}
\definecolor{MidGray}{gray}{0.8}
\definecolor{codegreen}{rgb}{0,0.6,0}
\definecolor{codegray}{rgb}{0.5,0.5,0.5}
\definecolor{codepurple}{rgb}{0.58,0,0.82}
\definecolor{backcolour}{rgb}{0.95,0.95,0.92}
\setminted[json]{frame=lines, framesep=2mm, baselinestretch=1.2, rulecolor=\color{black!80}, bgcolor=DarkGray}
\usemintedstyle[json]{monokai}
\setminted[apacheconf]{frame=lines, framesep=2mm, baselinestretch=1.2, rulecolor=\color{black!30}, bgcolor=LightGray}
\setminted[html+twig]{frame=lines, framesep=2mm, baselinestretch=1.2, rulecolor=\color{black!30}, bgcolor=LightGray}
\setminted[html+php]{frame=lines, framesep=2mm, baselinestretch=1.2, rulecolor=\color{black!30}, bgcolor=LightGray}

%\setlength{\parindent}{0px}  % Setea la indentacion de la primera linea de cada parrafo a cero pixeles.


\title{Fundamentos de Docker}
\author{@RuneCode}

\begin{document}
%% Portada
\includepdf{./portada/portada.pdf}

%% Clase 1
\section{Bienvenida al curso}%
Docker es una herramienta muy importante y su carrera como desarrolladores les
va a servir mucho. Docker como herramienta puede ahorrarte mucho trabajo
tedioso. Lo mejor de todo es que Docker es open source, no tiene ningún costo y
les hará la vida mucho más simple, por supuesto no es la herramienta para dar
los primeros pasos en programación. Con Docker ahora una persona puede hacer lo
mismo que talvés hace 10 años se necesitarían 3 o 4 personas.

%% Clase 2
\section{Problemáticas del desarrollo de software profesional}%
Para entender por qué nos va a servir Docker, es necesario entender su
implicancia, porque hoy en día está en prácticamente en todos los servidores de
los nuevos productos que se están haciendo y de los productos más grandes por
las ventajas que ofrece.\\

A la hora de hacer aplicaciones y proyectos de software nos podemos encontrar
con varios problemas, estos problemas los podemos agrupar en tres categorías:\\

\begin{itemize}
  \item Construir
  \item Distribuir
  \item Ejecutar.
\end{itemize}

Problemas al construir:\\
\begin{itemize}
  \item Dependencias de desarrollo
  \item Versiones de entornos de ejecución
  \item Equivalencia de entornos de desarrollo
  \item Equivalencia con entornos productivos
  \item Versiones / compatibilidad 3rd party
\end{itemize}

Problemas al distribuir:\\
\begin{itemize}
  \item Output de build heterogéneo.
  \item Acceso a servidores productivos.
  \item Ejecución nativa vs. virtualizada
  \item Serverless
\end{itemize}

Problemas al ejecutar:\\
\begin{itemize}
  \item Dependencias de aplicación
  \item Compatibilidad de sistema operativo
  \item Disponibilidad de servicios externos
  \item Recurso de hardware
\end{itemize}

Observando todos estos problemas que existen a la hora de pensar el software
más allá de lo que es escribir el código y nada más, es muy dificil encontrar
una solución simple, excepto que usar Docker, porque Docker lo que promete y lo
que logra es que puedas construir, distribuir y ejecutar tu código en cualquier
lado sin problemas.\\

Docker promete ser la solución a todo nuestros problemas de una manera simple y
sencilla.

%% Clase 3
\section{Qué es Docker: containarization vs virtualization}%
Cuando desarrollamos software, hay siempre problemas típicos y siempre están el
ámbito o de la construcción o de la distribución o de la ejecución de nuestro
código. Docker lo que nos permite es poder resolver todos esos problemas de
manera muy simple.\\

\textbf{El problema}: 

%% Clase 4
\section{Instalación de Docker}%
Vamos a ver cómo podemos instalar Docker en diferentes sistemas operativos,
tendrás el enlace en los archivos de esta clase.\\

\begin{itemize}
  \item Para usuarios Windows pueden la documentación en este
    \href{https://docs.docker.com/docker-for-windows/}{enlace}.
  \item Para usuarios Linux, este es el
    \href{https://docs.docker.com/install/linux/docker-ce/ubuntu/}{enlace}:
  \item Para usuarios Mac, este es el
    \href{https://docs.docker.com/docker-for-mac/}{enlace}.
\end{itemize}

Puedes comprobar que todo está funcionando, ingresa el comando \textbf{docker}
en la terminal.

%% Clase 5
\section{Primeros pasos: Hola mundo y Docker Engine}%
Realicemos nuestro primer 'Hola Mundo' en Docker utilizando el comando:\\

\begin{minted}{bash}
  docker run hello-world
\end{minted}

La arquitectura de Docker funciona cliente - servidor y además utiliza 'daemon'
al conectarse con los contenedores.

%% Clase 6
\section{Contenedores}%
Los contenedores son el concepto fundamentalal hablar de docker. Un contenedor
es una entidad lógica, una agrupación de procesos que se ejecutan de forma
nativa como cualquier otra aplicación en la máquina host.\\

Un contenedor ejecuta sus procesos de forma nativa.\\


%% Clase 7
\section{Explorar el estado de docker}%
Vamos a conocer algunos trucos de nuestra consola para explorar el estado del
docker\\

Para listar todos los contenedores de Docker, utilizamos el comando:\\

\begin{minted}{bash}
  docker ps -a
\end{minted}

Podemos inspeccionar un contenedor en específico utilizando:\\

\begin{minted}{bash}
  docker inspect nombreDelContenedor
\end{minted}

%% Clase 8
\section{El modo interactivo}%


%% Clase 9
\section{Ciclo de vida de un contenedor}%


%% Clase 10
\section{Exponiendo contenedores al mundo exterior}%
Los contenedores están aislados del sistema y a nivel de red, cada contenedor
tiene su propia stack de net y sus propios puertos.\\

Debemos redirigir los puertos del contenedor a los de la computadora y lo
podemos hacer al utilizar este comando:

\begin{minted}{bash}
  docker run -d --name server -p 8080:00  nombreDelContenedor
\end{minted}


%% Clase 11
\section{Datos en Docker}%

%% Clase 12
\section{Datos con Docker: Volumes}%
A pesar de que no es lo más divertido que tiene Docker, esta herramienta nos
permite recuperar datos que podíamos dar por perdido.\\

Existen tres maneras de hacer permanencia de datos:

\begin{itemize}
  \item Bind mount
  \item Volume
  \item tmpfs mount
\end{itemize}

%% Clase 13
\section{Concepto fundamental de Docker: imágenes}%
Las \textbf{imágenes} son un componente fundamental de Docker y sin ellas los
contenedores no tendrían sentido. Estas imágenes son fundamentalmente
plantillas o templates.\\

Algo que debemos tener en cuenta es que las imágenes no van a cambiar, es
decir, una vez este realizada no la podremos cambiar.\\


%% Clase 14
\section{Construyendo nuestras propias imágenes}%
Vamos a crear nuestras propias imágenes, necesitamos un archivo llamado
DockerFile que es la "receta" que utiliza Docker para crear imágenes.\\

Es importante que el DockerFile siempre empiece con un "FROM" sino, no va a
funcionar.\\

El flujo para construir en Docker siempre es así:
Dockerfile $–$ build $\rightarrow$ Imágen $–$ run $\rightarrow$ Contenedor


%% Clase 15
\section{Comprendiendo el sistema de capas}%


%% Clase 16
\section{Usando docker para desarrollar aplicaciones}%
En esta clase vamos a aplicar lo aprendido, y empezar a desarrollar con docker
utlizando como base un proyecto desarrollado en node cuyo enlace lo puedes
encontrar en los archivos del curso.\\

%% Clase 17
\section{Reto}%
Pon a prueba tus conocimientos


%% Clase 18
\section{Entendiendo el cache de layers para estructurar correctamente tus imágenes}%

%% Clase 19
\section{Docker networking: colaboración entre contenedores}%
Podemos conectar 2 contenedores de una manera fácil sencilla y rapida. Vamos a
ver que con tan solo unos comandos tendremos la colaboración entre contenedores.\\


%% Clase 20
\section{Docker-compose: la herramienta todo en uno}%
Docker compose es una herramienta que nos permite describir de forma
declarativa la arquitectura de nuestra aplicación, utiliza composefile
(docker-compose.yml).\\

%% Clase 21
\section{Trabajando con docker-compose}%


%% Clase 22
\section{Docker-compose como herramienta de desarrollo}%


%% Clase 23
\section{Conceptos para imágenes productivas}%


%% Clase 24
\section{Manejando docker desde un contenedor}%
Vamos en esta clase a ver uno de los secretos mejor guardados de docker. Vamos
a ver como manejar un contenedor desde otro contenedor. Es una manera muy
interesante para aprovechar la potencia de docker y todas las ventajas que
ofrece.\\


%% Clase 25
\section{Cierre del curso}%
Te invitamos a tomar el examen y poner a prueba tus conocimientos.


















\end{document}

