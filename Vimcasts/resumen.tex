%%
%% (
%%  )\ )                             (
%%  (()/(   (            (             )\  )   (
%%   /(_))  ))\   (       ))\  (   (   (()/(   ))\
%%   (_))  /((_)  )\  )  /((_) )\  )\   ((_))/((_)
%%   | _ \(_))(  _(_/( (_) )  ((_)((_)  _| |(_))
%%   |   /| || || ' \))/ -_)/ _|/ _ \/ _` |/ -_)
%%   |_|_\ \_,_||_||_| \___|\__|\___/\__,_|\___|
%%

\documentclass{article}
\usepackage[utf8x]{inputenc}
\usepackage{amsmath}
%\usepackage{slashbox}
\usepackage{amsfonts}
\usepackage{amssymb}
\usepackage{graphicx} % Paquete para incluir imágenes en el documento LaTeX
\usepackage{hyperref}
\hypersetup{
  colorlinks=true,
  linkcolor=blue,
  filecolor=magenta,
  urlcolor=cyan,
}
\urlstyle{same}
\usepackage{varwidth}

\newcommand\tab[1][1cm]{\hspace*{#1}}

\usepackage{multirow}

\usepackage[a4paper,rmargin=1.5cm,lmargin=1.5cm,top=1.5cm,bottom=1.5cm]{geometry}

\usepackage{pdfpages}

\usepackage{xcolor}
\usepackage{minted}
\setminted[cpp]{frame=lines, framesep=2mm, baselinestretch=1.2, rulecolor=\color{black!80},
                bgcolor=DarkGray,fontsize=\normalsize}
\usemintedstyle[cpp]{monokai}
\setminted[python]{frame=lines, framesep=2mm, baselinestretch=1.2, rulecolor=\color{black!80}, bgcolor=DarkGray}
\usemintedstyle[python]{monokai}
\setminted[vim]{frame=lines, framesep=2mm, baselinestretch=1.2, rulecolor=\color{black!30}, bgcolor=LightGray}
\setminted[javascript]{frame=lines, framesep=2mm, baselinestretch=1.2, rulecolor=\color{black!80}, bgcolor=DarkGray}
\usemintedstyle[javascript]{monokai}
\setminted[php]{frame=lines, framesep=2mm, baselinestretch=1.2, rulecolor=\color{black!30}, bgcolor=LightGray}
\setminted[html]{frame=lines, framesep=2mm, baselinestretch=1.2, rulecolor=\color{black!30}, bgcolor=LightGray}
\setminted[bash]{baselinestretch=1.2,rulecolor=\color{black!30},fontsize=\footnotesize,bgcolor=LightGray}
\definecolor{LightGray}{gray}{0.98}
\definecolor{DarkGray}{gray}{0.1}
\definecolor{MidGray}{gray}{0.8}
\definecolor{codegreen}{rgb}{0,0.6,0}
\definecolor{codegray}{rgb}{0.5,0.5,0.5}
\definecolor{codepurple}{rgb}{0.58,0,0.82}
\definecolor{backcolour}{rgb}{0.95,0.95,0.92}

\setlength{\parindent}{0px}  % Setea la indentacion de la primera linea de cada parrafo a cero pixeles.


\DeclareUnicodeCharacter{2014}{\dash}

\title{Introducción a JavaSE}
\author{@RuneCode}

\begin{document}
%% Portada
\includepdf{./portada/portada.pdf}

%% Screencast 1
\section{Mostrar invisibles}%
La función \textbf{list} de Vim se puede usar para revelar caracteres ocultos,
como tabulaciones y líneas nuevas. En este episodio, vamos a demostrar cómo
personalizar la apariencia de estos caracteres modificando la configuración de
listchars. Continúo mostrando cómo hacer que estos caracteres invisibles se
mezclen con su colortheme.\\

TextMate tiene una opción para \textbf{Mostrar invisibles}, que le permite ver
los caracteres de tabulación y final de línea. Cuando esto está habilitado, los
caracteres de tabulación aparecen como un pequeño triángulo, y el final de las
líneas aparece como un 'signo no' (¬). Esto es útil para distinguir entre
pestañas y espacios, y para revelar espacios finales al final de una línea.En
Vim, podemos mostrar caracteres invisibles habilitando \texttt{:set list}.
Puede ocultar estos caracteres nuevamente al ejecutarlos \texttt{:set nolist},
o puede alternar entre mostrarlos y ocultarlos al ejecutarlos  \texttt{:set
list!}. Si desea poder hacer esto rápidamente, puede asignar el comando de
alternancia a algo más conveniente.  Intente poner lo siguiente en su .vimrc:

\begin{minted}{vim}
  nmap <leader> l: set list! <CR>
\end{minted}

Ahora puede alternar entre mostrar y ocultar caracteres invisibles con
\textbf{{\textbackslash}l}. Tenga en cuenta que si ha configurado su mapleader
en un valor distinto de la barra diagonal inversa, deberá usarlo en su lugar.\\

\textbf{Personalizar símbolos}\\
Por defecto, el carácter de tabulación se representa literalmente como \^ I, y
el final de las líneas se representa con un signo de dólar.\\

Podemos personalizar los símbolos utilizados para representar caracteres
invisibles cambiando la listcharsconfiguración. Si desea utilizar los mismos
símbolos que TextMate, coloque lo siguiente en su .vimrc:

\begin{minted}{vim}
  set listchars=tab:▸\ ,eol:¬
\end{minted}

Si lo desea, puede personalizar otros caracteres invisibles además de las
pestañas y el final de las líneas. Para obtener más información, obtenga ayuda
sobre listchars:

\begin{minted}{vim}
  :help listchars
\end{minted}


%% Screencast 2









\vspace{2cm}
\LARGE\textit{RuneCode}


\end{document}

